% this file declares acronyms. There are two classes: abbrev for abbreviations and nomencl for nomenclature

\acsetup{
  first-style=short
}

% class `abbrev': abbreviations:
\DeclareAcronym{CE}{
  short = CE ,
  long  = contractile element ,
  class = abbrev
}
\DeclareAcronym{SEE}{
  short = SEE ,
  long  = serial elastic element ,
  class = abbrev
}
\DeclareAcronym{PEE}{
  short = PEE ,
  long  = parallel elastic element ,
  class = abbrev
}
\DeclareAcronym{SDE}{
  short = SDE ,
  long  = serial damping element ,
  class = abbrev
}
\DeclareAcronym{MTU}{
  short = MTU ,
  long  = muscle tendon unit ,
  class = abbrev
}

% class `nomencl': nomenclature
\DeclareAcronym{angelsperarea}{
  short = \ensuremath{a} ,
  long  = The number of angels per unit area ,
  sort  = a ,
  class = nomencl
}
\DeclareAcronym{numofangels}{
  short = \ensuremath{N} ,
  long  = The number of angels per needle point ,
  sort  = N ,
  class = nomencl
}
\DeclareAcronym{areaofneedle}{
  short = \ensuremath{A} ,
  long  = The area of the needle point ,
  sort  = A ,
  class = nomencl
}

% usage: \ac{ny}, \ac{la} and \ac{un} are abbreviations 
%        whereas \ac{angelsperarea}, \ac{numofangels} and \ac{areaofneedle} are part of the nomenclature
