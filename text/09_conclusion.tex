
\chapter{Conclusion and Future Work}\label{sec:conclusion_and_future_work}
To conclude this work, we summarize the presented models, algorithms, implementations, studies and the main findings. Moreover, we given an outlook on future work and additional research questions that can be approached building on our work.

\section{Summary of the Work}

The present work consisted of the following parts: After the introduction in \cref{chap:introduction}, we compared two modeling approaches to describe the movement of the upper arm in \cref{chap:comparative_study}. Based on data of experimental trials we conducted during a graduate school workshop, we developed a first, data-driven model using Gaussian process regression and a second model based on a biophysical simulation with two muscle models. The parameters for the biophysical simulation were fitted to experimental training data using numeric optimization. The comparison of the two approaches revealed a slightly better fit for the biophysical simulation model. This approach has the additional benefit of biophysical insight into the functionng of the system and provides estimates for subject-specific muscle parameters. While this study used Hill-type muscle models, which describe muscle forces on a 1D line of action, we considered more accurate multi-scale models in the remainder of this work.

This work established a computational framework for multi-scale modeling of skeletal muscles, their neural activation, contraction and generated EMG signals on the skin surface. Our approach combined existing models of various parts of the neuromuscular system into a comprehensive multi-scale model framework.

\Cref{sec:generation_of_meshes_for_multiscale} dealed with the generation of structured 3D meshes and embedded 1D meshes for muscle fibers. The approach of only using structured meshes, which allow for a simple domain decomposition proved to be beneficial for the parallel performance of the simulations.  
We described a workflow how to obtain these meshes from biomedical imaging data. We developed a serial and a parallel algorithm to construct the required meshes and ensure a good mesh quality, even for highly-resolved meshes. The algorithms were based on our novel technique to use harmonic maps to transform reference meshes to cross sectional slices of the muscle mesh.

In \cref{sec:muscle_fibers_and_motor_units}, we described ways to associate muscle fibers with motor units (MUs) in a physiological manner. We developed efficient algorithms for this task for different premises, and employed the algorithms to associate up to \num{270000} muscle fibers to \num{100} MUs for the subsequent use in our simulations.

In \cref{chap:models_and_discretization}, we first described all equations of the state-of-the models that we used and how they can be combined into a multi-scale description. Then, we described their discretization using the Finite Element method for the spatial derivative terms and various timestepping and operator splitting schemes for the temporal derivatives. One original contribution is the derivation of the Finite-Element formulation for the multidomain equation. Further, we gave a detailed description of the nonlinear solid mechanics discretization, which we used in our implementation.

Next, we presented details on our simulation software OpenDiHu, which we use to solve various combinations of the described multi-scale model framework to simulate the neuromuscular system. \Cref{chap:usage} gave an introduction to the design and usage of the software and demonstrated its application using various example problems.

\Cref{sec:implementation} described the implementation of OpenDiHu in more detail, motivated various design decisions, introduced the data handling, several algorithms, e.g., to construct a parallel domain decomposition or to map data between meshes, and described the implementation of various solvers for particular parts of the multi-scale model.

\Cref{sec:results} presented numerical results, which were obtained using our simulation software. We simulated the passive mechanical behavior of muscle tissue, subcellular models given in CellML description, electrophysiology on muscle fibers, electric conduction in the muscle and the adipose tissue to obtain surface EMG signals, electrophysiology using the 3D homogenized multidomain description, and coupled scenarios of electrophysiology and muscle contraction. We discussed effects of model and structural parameters and interpreted the obtained simulation results.

In \cref{sec:performance_analysis}, we analyzed the computational performance of our software in general and various solvers in particular. We conducted numerical studies on universal convergence properties with the software OpenCMISS, whose insights could also help to parameterize the numerical solvers in OpenDiHu. Studys on mesh widths and used linear solvers were also carried out directly using OpenDiHu. We evaluated various optimization options in OpenDiHu and compared the most optimized settings in OpenDiHu with the baseline solver OpenCMISS, yielding a high speedup of over two orders of magnitude. Moreover, we investigated the computational performance of our models on the GPU, and conducted parallel strong scaling and parallel weak scaling tests on small clusters and the supercomputers at the High Performance Computing Center Stuttgart.

\section{Summary of Main Findings}

The present work simulated numerous scenarios with various different model combinations, which each revealed different insights. In the following, we summarize the observed findings, regarding biomechanical observations in \cref{sec:observations_biophysics} and performance measurements in \cref{sec:observations_performance}.

\subsection{Observations from the Fields of Biophysics and Biomechanics}\label{sec:observations_biophysics}
The comparison of the linear and nonlinear mechanics models in \cref{sec:solver_solid_mechanics} showed qualitatively different results and demonstrated, that the accurate behavior of deforming muscle tissue can only be described by a proper nonlinear anisotropic solid mechanics model. 

The accuracy of the simulated EMG signals depending on the number of fibers and mesh resolution was previously also an open question. Our numerical studies in \cref{sec:action_potential_velocity}, which compared the action propagation velocity showed that a 1D mesh width for the muscle fibers of \SI{100}{\micro\meter} or 100 elements per \SI{}{\centi\meter} gives reasonably accurate results. 

To evaluate the 3D mesh width and the spacing between muscle fibers, we conducted simulations with different 3D mesh resolutions and numbers of fibers in \cref{sec:effects_of_the_mesh_width_emg}. The number of fibers was scaled up to the realistic number of \num{270000} fibers in a biceps brachii muscle. We concluded that the most accurate solution uses a mesh width that is as fine as possible, as the qualitative EMG results were different for every refinement step. This emphasizes the need for highly resolved simulation scenarios for realistic EMG computations and the need for High Performance Computing techniques.

However, if the EMG is to be sampled by electrodes, i.e., if the EMG recording process should also be part of the simulation, lower mesh widths are possible, as the EMG is only captured at the locations of the electrodes.

Another possible approach to reduce the computational effort for EMG simulations would be to only consider the muscle tissue down to a certain depth below the surface with the EMG electrodes. It is known that the EMG signal is highly influenced by MUs whose territories are located close to the electrodes. We reproduced this effect in \cref{sec:simfiber_mu}. However, our studies in \cref{sec:simfiber_decomposition} with EMG decomposition algorithms showed that large MUs located opposite to the EMG electrodes at the deepest muscle tissue layers are detectable in the surface EMG signals. Thus, neglecting the deeper parts of the muscle would remove relevant information from the system.

Further, the smoothing effect of a layer of adipose tissue on top of the muscle on EMG recordings was found in our simulations, both with the fiber based approach in \cref{sec:simfiber_fat} and with the multidomain approach in \cref{sec:multidomain_components,sec:multidomain_simulation_emg}. One advantage of our simulations compared to experimental studies is that the thickness of the fat layer is known exactly and can also be adjusted.

Simulations of muscle contraction with coupled electrophysiology and solid mechanics models showed a spatially inhomogeneous contraction  for the biceps muscle while the muscle activation is ramped up. The simulation in \cref{sec:fiber_based_contraction} of an isolated, contracting muscle belly without tendons showed transverse bending alternating to the left and right hand sides, due to the subsequently activated MUs at different sides of the muscle. We also simulated the biceps brachii muscle together with its tendons and observed a ripple in the generated muscle force, which is caused by the same inhomogeneous MU activity.

The simulations of muscle contraction also showed that the model can only achieve a maximum contraction of approximately \SI{85}{\percent}, if the muscle is initially in a stress-free state. However, muscles within the musculoskeletal system are known to exhibit presstresses even in their relaxed states. Accordingly, we added prestress to our simulations. The amount of prestress is adjustable in the simulation settings, and the required amount can be determined by a comparison with experimental studies.

% linear-nonlin
% accuracy -> high mesh resolution
% fibers: effects of fat mesh, distance between fibers to surface, size of MUs (EMG decomposition)
% multidomain: more smoothed
% coupled solid mechanics: inhomogenous contraction, prestretch required (without only 85% contraction)

\section{Summary of Performance Results}\label{sec:observations_performance}

A major part of the current work was also concerned with improving the performance of the simulation software, and, thus, enabling larger simulation scenarios in shorten runtimes.
Previously, literature on biophysical, multi-scale models of skeletal muscle was mainly focused on modelling and interpretation of the results, rather than targeting efficient computations. The work of Röhrle et al. \cite{Roehrle2012} introduced the multi-scale model, which we base our work on, and simulated the tibialis anterior muscle using a 3D mechanics mesh with 12 elements. The work of Heidlauf et al. \cite{Heidlauf2013} considered the same geometry and simulated 400 muscle fibers. The authors parallelized their OpenCMISS based implementation for a fixed number of four processes. We built upon this work with the goal to push the limits of feasible problem sizes, and, in \cref{sec:effects_of_the_mesh_width_emg}, executed our optimized simulation with \num{26912} processes, \num{273529} muscle fibers and a 3D mesh for the electrophysiology model with approximately \num{1e8} degrees of freedom.

The performance analyses in \cref{sec:performance_analysis} showed that the subcellular contributes a large portion of the total runtime and, thus, is the most crucial part to optimize. By using proper memory layouts, vectorization is possible. Our approach of using explicit vector instructions outperformed the auto-vectorization capabilities of the compiler. The approximation of the exponential function and an improved parallelization scheme for the 1D electric conduction problem additionally contributed to a high speedup. The comparison to the baseline solver OpenCMISS Iron in a strong scaling study in \cref{sec:strong_scaling_runtimes_opencmiss_opendihu} revealed a maximum speedup of 363 for the pure algorithmic improvements and additional speedups of 2.5, shown in \cref{sec:opencmiss_numeric_improvements}, by using more efficient numeric methods.

In addition, the memory characteristics of the solvers were investigated in \cref{sec:strong_scaling_runtimes_opencmiss_opendihu}. 
The linear increase in memory consumption of the baseline solver in a weak scaling setting was improved to a nearly constant scaling. Our analysis using a roofline performance model showed that our solvers are compute bound and achieve a computational performance of approximately \SI{25}{\percent} peak performance, which is a very high value.

Moreover, hybrid shared-distributed memory parallelism and computations on the GPU were investigated, but both found to be not competitive with our highly optimized distributed memory implementation. For the GPU, potentially more efficient approaches than our approach using OpenMP exist such that a performance improvement in the future could be possible.

The modularity of the CellML infrastructure, where computational models can be shared among researchers and are interchangeable in multi-scale simulations was preserved during all optimization endeavours. Our approach was to implement a source-to-source code generator, which transformed the given CellML code into optimized code for the CPU or the GPU.

For the solution of the multidomain model, we evaluated various preconditioners and selected the most performant one for our computations.
One previously unforeseen result is the large discrepancy of required runtime between the fiber based and the multidomain based electrophysiology models, presented in \cref{sec:solver_multidomain_model}. Due to the model structure, we measured longer computation times by a factor of 1000 for the multidomain model. However, the multidomain model is useful in practice as it can simulate effects that are not captured by the fiber based model. \Cref{sec:multidomain_differences} contains a detailed comparison.

In summary, we provide a computationally efficient and scalabe tool for applied biophysics researchers to solve problems in the domains of EMG generation and muscle contraction. For example, the effect of different muscle fiber organizations and MU recruitment strategies can be tested with our software. We demonstrated its use with state-of-the-art EMG decomposition algorithms, which provide the bridge to the experimental domain.
Thus, we hope to contribute one step on the pathway of complementing in-vivo with in-silico experiments to increase the understanding of the human neuromuscular system.

% provide a tool for applied biophysics researchers
% --------------------
% complement in-vivo and in-silico experiments
% test different muscle fiber organizations -> possible
% decomposition of EMG -> tested


% vc better than auto-vec, AVX-512, memory layout
% comparison to OpenCMISS
% preserve CellML modularity -> code generator
% GPU, OpenMP
% proper choice of solvers
% multidomain performance vs fibers


% performance
% --------------
% Röhrle2012: TA 12 elements, MU association
% Heidlauf2013: 400 fibers, TA, OpenCMISS, mechanics
% made new insights possible  


%wir hatten ja mal vorgenommen dass
%tatsächlich ist gelungen:
%überraschenderweise, dass: 

%bewerten mit den Ergebnissen
%was bewahrheitet, 

%unterschiede zwischen grob und hochaufgelöster Sim
%anzahl realitäts anzahl Muskelfaser anzahl

\section{Outlook and Future Work}\label{sec:future_work}
 
The presented work could be extended in multiple directions.
First, some ideas for further performance improvement could be implemented and evaluated. 
The monodomain equation could be solved with implicit-explicit (IMEX) schemes, which could potentially lead to a higher precision. 
The numerically stiff subcellular model is currently solved explicitly. Implicit schemes could be developed, which infer the iteration equations from the parsed model code in a preprocessing step. 

To improve the performance of the multidomain model, some algorithmic improvements are imaginable. 
The 3D problems of action potential propagation in the muscle volume for every compartment could be restricted to the subset of nodes, where the occupancy factors are above a ceratin threshold, effectively reducing the problem sizes, and reducing the effect of higher MU counts on the runtime. However, this would pose difficulties to ensure a balanced parallel domain decomposition.
Instead of the current parallel partitioning of the domain, the multidomain model could also be parallelized by distributing the MUs to different processes or by a combination of both approaches.

On the numeric side, an extended error analysis could be carried out for all model parts, and the timestep widths, which are currently chosen conservatively, could potentially be increased while keeping the numeric error below a given threshold. Error estimators could be developed, which would allow an adaptive adjustment of the timestep widths.
The 3D solvers used to solve the 3D electrophyisology and multidomain problems could be enhanced with geometric or algebraic multigrid preconditioners.

Since all subcellular points in a muscle usually are in similar states at any time, a middle course between using analytic descriptions of action potential propagation, as in \cref{sec:sim_rosenfalck}, and the normal numerical computation could be chosen and surrogate models could be adaptively added to the computational description.

On the technical side, computations on the GPU could be re-evaluated in the future using the existing OpenMP approach with more mature compiler versions or different accelerator targeting programming technologies.

Second, the range of simulated models could be extended. The simulations could be applied to different muscle geometries such as the triceps brachii or the tibialis anterior muscle. Muscles with more complex geometries and fiber arrangements could be investigated. 
A mechanically coupled problem of agonist-antagonist pairs could be considered, such as a system of biceps and triceps. Apart from the mechanical coupling, a coupling of the neural recruitment involving sensory organs in the muscles could be implemented and used to approach biomechanical research questions. Such a neuromuscular feedback loop could also be investigated first for a single muscle, e.g., by extending the preliminary implementation in OpenDiHu.

Pathological conditions could be simulated to understand muscular diseases and neuromuscular electrical stimulation of the muscle for stroke rehabilitation could be considered.

By using the preCICE adapters in OpenDiHu, more advanced mechanics solvers could be coupled to an electrophysiology simulation in OpenDiHu, allowing to study mechanical effects of surrounding tissue.

On a larger scale, the interplay of more organs could be taken into account. Blood perfusion and muscle metabolism could be added, and coupled by models of the lung and general metabolism in the organism. Thus, a digital human model can be envisioned, which allows to study the effects of anomalies and to develop new therapies, effectively utilizing simulation technology for human wellbeing.

%reduced to 1D problems using appropriate transformations.


% performance improvements
% ----------------------------
% more timestepping methods: CVODE (https://computing.llnl.gov/projects/sundials/cvode), imex
% different parallelisation where not all ranks have to be involved (for multidomain) -> this feature already exists for the fibers with multipleInstances
% more numeric tests on exp function? no
% multidomain: compute 3D problem as 1D problem, or adaptive computation of the parts of the fr factors
% GPU

% extensions
% -------------
% other muscles, more muscles,
% couple more models:
% neuromuscular feedback loop with sensors, 
% 






