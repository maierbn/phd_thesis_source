%\documentclass[fleqn,reqno,a4paper,parskip=half]{scrartcl}
\documentclass[fleqn,reqno,a4paper,parskip=half]{scrbook}
%\usepackage{showkeys}      % zeigt label-Bezeichner an

%%%%%%%%%%%%%%%%   Pakete   %%%%%%%%%%%%%%%%%%

\usepackage{charter} %Charter fuer englische Texte
%\linespread{1.05} % Durchschuss für Charter leicht erhöhen

%\usepackage[mathletters]{ucs} %direkt griechisches im Mathe modus
%\usepackage[utf8]{inputenc}
\usepackage[utf8]{inputenc}
%\usepackage[utf8x]{inputenc}
%\usepackage[T1]{fontenc}
%\usepackage{times}
\usepackage{pmboxdraw}

% utf8x is incompatible with biblatex, the following trick is a remedy
% source: https://tex.stackexchange.com/a/213177
\input{binhex}
\makeatletter
\def\uc@dclc#1#2#3{%
  \ifnum\pdfstrcmp{#2}{mathletters}=\z@
    \begingroup\edef\x{\endgroup
      \noexpand\DeclareUnicodeCharacter{\hex{#1}}}\x{#3}%
  \fi
}
\input{uni-3.def}
\def\uc@dclc#1#2#3{%
  \ifnum\pdfstrcmp{#2}{default}=\z@
    \begingroup\edef\x{\endgroup
      \noexpand\DeclareUnicodeCharacter{\hex{#1}}}\x{#3}%
  \fi
}
\input{uni-34.def}
\makeatother






%TODO_Lorin: besser?
   %\usepackage{uniinput}       % für Unicode-Zeichen, wird momentan nicht verwendet, deshalb auskommentiert by Benni


%\usepackage[ngerman]{babel}
%\usepackage{ngerman}
\usepackage[tbtags,sumlimits,intlimits,namelimits]{amsmath}

\usepackage{amsfonts}
\usepackage{amssymb}
\usepackage{bbm}
\usepackage{ulem}
\usepackage{tikz}
\usepackage{pgf}
\usepackage{ifpdf}
\usepackage{color}
\usepackage{esint}
\usepackage{framed}
\usepackage{wasysym}  % \ocircle
%\usepackage{calc}   % \begin{tabular}[b]{|p{\textwidth-2\tabcolsep}|}
\usepackage{makecell}  % \makecell{...\\...} to break in table cell
\usepackage{textcomp}
\usepackage{fancyvrb}  % improved verbatim 

%\usepackage[colorlinks=true,linkcolor=black,citecolor=black,urlcolor=black]{hyperref}  % print
\usepackage[colorlinks=true,linkcolor=blue,citecolor=blue,pdfpagelabels,plainpages=false]{hyperref}    % web
\usepackage[top=2.3cm, bottom=3.45cm, left=2.3cm, right=2.3cm]{geometry}
%\numberwithin{equation}{section}
\usepackage{chngcntr}
\counterwithin*{section}{part}
%\graphicspath{{images/png/}{images/}}        % Pfad, in dem sich Grafikdateien befinden
%\usepackage{subfigure}          % Unterbilder, deprecated
%\usepackage(subfig}

\usepackage[all]{hypcap}
%\usepackage{cite}           % Literatur
\usepackage{graphicx}       % Bilder in Tabellen
\usepackage{float}          % eigene Float-Umgebungen, H-Option, um Bilder an der aktuellen Stelle anzuzeigen
\usepackage{caption}
\usepackage{subcaption,array}
%\usepackage{array}

\restylefloat{figure}       % Bilder an der Stelle, wo sie eingebunden werden
\usepackage{multirow}
\usepackage{listings}       % Darstellung von Source-Code
\usepackage{framed}         % Rahmen um Text
\usepackage{arydshln}       % gestrichelte Linie in Tabelle mit \hdashline
\usepackage{longtable}     % Tabellen mit automatischen Seitenumbruch
\usepackage{layouts}        % textwidth in cm: \printinunitsof{cm}\prntlen{\textwidth}
 % \cref verbessert
\usepackage[
  % capitalize names automatically
  capitalise,
  % mark all of "Lemma 1" as a link, not only "1"
  nameinlink,
  % pass language used in document to enable cleveref in otherlanguage env
  ngerman,english,
]{cleveref}

% generate Eqs. instead of Equations
\crefmultiformat{equation}{#2Eqs.~(#1)#3}{ and~#2(#1)#3}{, #2(#1)#3}{ and~#2(#1)#3}

%\usepackage{ziffer}       % Komma in Dezimalzahlen
\usepackage{url}          % Links
%% end old
\definecolor{darkblue}{rgb}{0,0,.5}
\definecolor{black}{rgb}{0,0,0}

\usepackage[framemethod=TikZ]{mdframed}       % Rahmen um Text und Gleichungen


%\usepackage{arydshln}      % gestrichelte Linie in Tabelle mit \hdashline
\usepackage{dirtytalk}          % \say{...} erzeugt (deutsche) Anführungszeichen

\usepackage{tipa}
\usepackage{transparent}    % needed for inkscape generated pdf_tex files
\usepackage{multicol}       % multiple columns
\usepackage{moreverb}       % verbatimwrite
\usepackage{verbatimbox}    % \begin{verbbox}
\usepackage{booktabs}
\usepackage{morefloats}     % Increase the number of simultaneous LaTeX floats
\usepackage{enumitem}       % more control oover itemize
%\usepackage{algorithm2e}
%\usepackage{algorithmic}
%\usepackage{algpseudocode}

\newsavebox\lstbox
\mdfdefinestyle{MyFrame}{%
    innertopmargin=0pt,
    innerbottommargin=10pt,
    innerrightmargin=20pt,
    innerleftmargin=20pt}

\definecolor{darkgreen}{HTML}{009900}
    
% settings for algorithm
\lstset{literate=%
    {Ö}{{\"O}}1
    {Ä}{{\"A}}1
    {Ü}{{\"U}}1
    {ß}{{\ss}}1
    {ü}{{\"u}}1
    {ä}{{\"a}}1
    {ö}{{\"o}}1
    {⇐}{{$\leftarrow$}}1
    {>=}{{$\geq$}}1
    {~}{{\textasciitilde}}1
    {`}{\textquotedbl}1,  
  language=C++,
  numbers=none,
  numberstyle=\tiny,
  xleftmargin=2.0ex, 
  %basicstyle=\small, %  print  whole  listing  small
  basicstyle=\small\ttfamily,
  morekeywords={elif,do,end,then,proc,local,Eingabe,Ausgabe,alignof,loop,each},
  deletekeywords={new},
  columns=flexible,   % alignment
  tabsize=2,    % size of tabs
  keepspaces,
  gobble=2,    % remove 2 characters at begin of each line
  mathescape    % wandle $$ in latex um
}

% Versuche stärker, Abbildungen dort einzubinden, wo sie definiert wurden
\renewcommand{\topfraction}{.85}      % Anteil, den floats auf einer Seite von oben her einnehmen dürfen
\renewcommand{\bottomfraction}{.7}    % Anteil, den floats auf einer Seite von unten her einnehmen dürfen
\renewcommand{\textfraction}{.15}       % Anteil der Seite, der mind. für Text zur Verfügung steht
\renewcommand{\floatpagefraction}{.66}  % Anteil der Seite, der belegt sein muss, bevor eine weitere Seite angelegt wird
\setcounter{topnumber}{9}               % maximale Anzahl floats, die im oberen Bereich der Seite sein dürfen
\setcounter{bottomnumber}{9}            % maximale Anzahl floats, die im unteren Bereich der Seite sein dürfen
    

\usepackage{charter} %Charter fuer englische Texte
%\linespread{1.05} % Durchschuss für Charter leicht erhöhen

\usepackage[utf8]{inputenc}
\usepackage{pmboxdraw}

% utf8x is incompatible with biblatex, the following trick is a remedy
% source: https://tex.stackexchange.com/a/213177
\input{binhex}
\makeatletter
\def\uc@dclc#1#2#3{%
  \ifnum\pdfstrcmp{#2}{mathletters}=\z@
    \begingroup\edef\x{\endgroup
      \noexpand\DeclareUnicodeCharacter{\hex{#1}}}\x{#3}%
  \fi
}
\input{uni-3.def}
\def\uc@dclc#1#2#3{%
  \ifnum\pdfstrcmp{#2}{default}=\z@
    \begingroup\edef\x{\endgroup
      \noexpand\DeclareUnicodeCharacter{\hex{#1}}}\x{#3}%
  \fi
}
\input{uni-34.def}
\makeatother

\usepackage[tbtags,sumlimits,intlimits,namelimits]{amsmath}

\usepackage{amsfonts}
\usepackage{amssymb}
\usepackage{bbm}
\usepackage{ulem}
\usepackage{tikz}
\usepackage{pgf}
\usepackage{ifpdf}
\usepackage{color}
\usepackage{esint}
\usepackage{framed}
\usepackage{wasysym}  % \ocircle
\usepackage{makecell}  % \makecell{...\\...} to break in table cell
\usepackage{textcomp}
\usepackage{fancyvrb}  % improved verbatim 


%\usepackage[colorlinks=true,linkcolor=black,citecolor=black,urlcolor=black]{hyperref}  % print
%\usepackage[colorlinks=true,linkcolor=blue,citecolor=blue,pdfpagelabels,plainpages=false]{hyperref}    % web
%\usepackage{hyperref}    % plain
\usepackage[top=2.3cm, bottom=3.45cm, left=2.3cm, right=2.3cm]{geometry}
%\numberwithin{equation}{section}
\usepackage{chngcntr}
\counterwithin*{section}{part}
%\graphicspath{{images/png/}{images/}}        % Pfad, in dem sich Grafikdateien befinden
%\usepackage{subfigure}          % Unterbilder, deprecated
%\usepackage(subfig}

%\usepackage{cite}           % Literatur
\usepackage{graphicx}       % Bilder in Tabellen
\usepackage{float}          % eigene Float-Umgebungen, H-Option, um Bilder an der aktuellen Stelle anzuzeigen
\usepackage{caption}
\usepackage{subcaption,array}
%\usepackage{array}

\restylefloat{figure}       % Bilder an der Stelle, wo sie eingebunden werden
\usepackage{multirow}
\usepackage{listings}       % Darstellung von Source-Code
\usepackage{framed}         % Rahmen um Text
\usepackage{arydshln}       % gestrichelte Linie in Tabelle mit \hdashline
\usepackage{longtable}     % Tabellen mit automatischen Seitenumbruch
\usepackage{layouts}        % textwidth in cm: \printinunitsof{cm}\prntlen{\textwidth}
 
%\usepackage{ziffer}       % Komma in Dezimalzahlen
\usepackage{url}          % Links
%% end old
\definecolor{darkblue}{rgb}{0,0,.5}
\definecolor{black}{rgb}{0,0,0}

\usepackage[framemethod=TikZ]{mdframed}       % Rahmen um Text und Gleichungen


%\usepackage{arydshln}      % gestrichelte Linie in Tabelle mit \hdashline
\usepackage{dirtytalk}          % \say{...} erzeugt (deutsche) Anführungszeichen

\usepackage{tipa}
\usepackage{transparent}    % needed for inkscape generated pdf_tex files
\usepackage{multicol}       % multiple columns
\usepackage{moreverb}       % verbatimwrite
\usepackage{verbatimbox}    % \begin{verbbox}
\usepackage{booktabs}
\usepackage{morefloats}     % Increase the number of simultaneous LaTeX floats
\usepackage{enumitem}       % more control oover itemize
%\usepackage{algorithm2e}
%\usepackage{algorithmic}
%\usepackage{algpseudocode}

\newsavebox\lstbox
\mdfdefinestyle{MyFrame}{%
    innertopmargin=0pt,
    innerbottommargin=10pt,
    innerrightmargin=20pt,
    innerleftmargin=20pt}

\definecolor{darkgreen}{HTML}{009900}
    
% settings for algorithm
\lstset{literate=%
    {Ö}{{\"O}}1
    {Ä}{{\"A}}1
    {Ü}{{\"U}}1
    {ß}{{\ss}}1
    {ü}{{\"u}}1
    {ä}{{\"a}}1
    {ö}{{\"o}}1
    {⇐}{{$\leftarrow$}}1
    {>=}{{$\geq$}}1
    {~}{{\textasciitilde}}1
    {`}{\textquotedbl}1,  
  language=C++,
  numbers=none,
  numberstyle=\tiny,
  xleftmargin=2.0ex, 
  %basicstyle=\small, %  print  whole  listing  small
  basicstyle=\small\ttfamily,
  morekeywords={elif,do,end,then,proc,local,Eingabe,Ausgabe,alignof,loop,each},
  deletekeywords={new},
  columns=flexible,   % alignment
  tabsize=2,    % size of tabs
  keepspaces,
  gobble=2,    % remove 2 characters at begin of each line
  mathescape    % wandle $$ in latex um
}

% Versuche stärker, Abbildungen dort einzubinden, wo sie definiert wurden
\renewcommand{\topfraction}{.85}      % Anteil, den floats auf einer Seite von oben her einnehmen dürfen
\renewcommand{\bottomfraction}{.7}    % Anteil, den floats auf einer Seite von unten her einnehmen dürfen
\renewcommand{\textfraction}{.15}       % Anteil der Seite, der mind. für Text zur Verfügung steht
\renewcommand{\floatpagefraction}{.66}  % Anteil der Seite, der belegt sein muss, bevor eine weitere Seite angelegt wird
\setcounter{topnumber}{9}               % maximale Anzahl floats, die im oberen Bereich der Seite sein dürfen
\setcounter{bottomnumber}{9}            % maximale Anzahl floats, die im unteren Bereich der Seite sein dürfen
    
% language-specific things (hyphenation, ...)
\usepackage[ngerman,main=american]{babel}

% biblatex package wants to have csquotes (otherwise: warning)
\usepackage{csquotes}

% lower-case page header
\usepackage[markcase=upper]{scrlayer-scrpage}

% dummy texts
\usepackage[math]{blindtext}

% amsmath with improvements
\usepackage{mathtools}

% math theorems
%\usepackage{amsthm}

% format math theorems
\usepackage{thmtools}

% for setting size of text area
\usepackage{geometry}

% for \luadirect command
%\usepackage{luacode}

% check mode
%\iftoggle{checkMode}{
  % show non-whitelisted hyphenations
%  \usepackage[mark]{lua-check-hyphen}
%}{}

% debug mode
%\iftoggle{debugMode}{
%  % show boxes, glues, and kerning
%  \usepackage{lua-visual-debug}
%}{}

% use GoMono as typewriter font
\usepackage[
  % otherwise, bold is not available ("undefined font shape" warnings)
  type1,
  % scale to match main font size
  scale=0.88,
]{GoMono}

% use TeX Gyre Heros as sans-serif font
\usepackage{tgheros}

% need T1 font encoding for Charter,
% otherwise there will be "undefined font shape" warnings
\usepackage[T1]{fontenc}

% use Bitstream Charter as main font
\usepackage[bitstream-charter]{mathdesign}

% do not use mathcal of mathdesign, but replace by standard mathcal
\DeclareSymbolFont{usualmathcal}{OMS}{cmsy}{m}{n}
\DeclareSymbolFontAlphabet{\mathcal}{usualmathcal}


% micro-typographic adjustments
\usepackage[
  % don't deactivate extensions due to document class draft option
  final,
  % increase letter spacing in small-caps
  tracking=smallcaps,
  % font expansion: don't stretch/shrink lines by more than 1%,
  % default of 2% looks a bit weird
  stretch=10,
  shrink=10,
]{microtype}

% reference last page of document
\usepackage{lastpage}

% extract number from page reference
\usepackage{refcount}

% graphics
\usepackage{graphicx}

% colors, load colortbl for coloring table rows
%\usepackage[table]{xcolor}

% contours around text
\usepackage{contour}

% additional space in table cells
\usepackage{cellspace}

% custom float captions, subfigures
\usepackage{subcaption}

% allows to put float caption beside image
\usepackage{sidecap}

% L, C, R column types for tables
\usepackage{array}

% X column type for tables (filling rest of line)
\usepackage{tabularx}

% more beautiful tables
\usepackage{booktabs}

% table cells spanning multiple rows
\usepackage{multirow}

% for controlling space in enumerate and itemize
\usepackage{enumitem}

% disable "Column type for cellspace package moved to 'C'." warning
% when loading siunitx
% (this is due to the cellspace package loaded above)
\usepackage{expl3}
\ExplSyntaxOn
\msg_redirect_name:nnn{siunitx}{moved-cellspace-column}{none}
\ExplSyntaxOff

% consistent style of units and numbers
\usepackage[binary-units=true]{siunitx}

% watermarks
\usepackage{everypage}

% rules left of theorems
%\usepackage[framemethod=tikz]{mdframed}

% drawings
\usepackage{tikz}

% styling of table of contents
\usepackage{tocbasic}

% local table of contents (mini TOC)
\usepackage{etoc}

% wrap text around boxes
\usepackage{wrapfig}

% initials/dropped captial letters at beginning of chapters
\usepackage{lettrine}

% inclusion of external PDFs
\usepackage{pdfpages}

% reference current chapter, section, subsection, ...
\usepackage{nameref}

% line spacing (one and a half lines)
\usepackage[onehalfspacing]{setspace}

% filter specific warnings
%\usepackage{silence}

% patch hard-coded stuff in packages
\usepackage{regexpatch}

% allow breaking after hyphens in long URLs
%\usepackage[hyphens]{url}

% PDF links
\usepackage{hyperref}

\usepackage[all]{hypcap}

% \cref verbessert
\usepackage[
  % capitalize names automatically
  capitalise,
  % mark all of "Lemma 1" as a link, not only "1"
  nameinlink,
  % pass language used in document to enable cleveref in otherlanguage env
  ngerman,english,
]{cleveref}

% generate Eqs. instead of Equations
\crefmultiformat{equation}{#2Eqs.~(#1)#3}{ and~#2(#1)#3}{, #2(#1)#3}{ and~#2(#1)#3}
%\crefrangemultiformat{equation}{#2Eqs.~(#1)#3}{ and~#2(#1)#3}{, #2(#1)#3}{ and~#2(#1)#3}
%\crefrangemultiformat{equation}
%  {Eqs. #3(#1)#4 to #5(#2)#6}
%  { and #3(#1)#4 to #5(#2)#6}
%  {, #3(#1)#4 to #5(#2)#6}
%  { and #3(#1)#4 to #5(#2)#6}

% algorithms
\usepackage{algorithm}
\usepackage{algorithmicx}

% pseudo-code
\usepackage[noend]{algpseudocode}

% create custom PDF bookmarks
\usepackage{bookmark}

% old formatting using the titlesec package, which is now obsolete
%\usepackage{titlesec} % space before chapter head
%\titleformat{\chapter}[display]{\normalfont\huge\bfseries}{\chaptertitlename\ \thechapter}{20pt}{\Huge}
\renewcommand*{\chapterformat}{\normalfont\large\bfseries Chapter~\thechapter\autodot\enskip}
%Chapter~\fontsize{60}{68}\selectfont\color{gray} \thechapter\autodot\enskip}

\renewcommand\chapterlinesformat[3]{\ifstr{#2}{}{}{#2\vspace*{5mm}\\*}{\Huge #3}}

% analogous to \cleardoublepage but for left page
\newcommand*\cleartoleftpage{%
  \clearpage
  \ifodd\value{page}\hbox{}\newpage\fi
}

% BibLaTeX
\usepackage[
  % abbreviate author names
  giveninits=true,
  % only show years in dates
  date=year,
  % use alphabetic style instead of numeric
  style=alphabetic,
  % only use first author's name for the style
  maxalphanames=1,
]{biblatex}

%\usepackage{scrhack}

% named references
%\usepackage[
%  % capitalize names automatically
%  capitalise,
%  % mark all of "Lemma 1" as a link, not only "1"
%  nameinlink,
%  % pass language used in document to enable cleveref in otherlanguage env
%  ngerman,english,
%]{cleveref}

\makeatletter

% ======================================================================
% Meta-Data
% ======================================================================

% meta-data variables
\newcommand*{\thetitle}{%
  Scalable Biophysical Simulations\texorpdfstring{\\}{}
  of the Neuromuscular System%
}
\newcommand*{\theapproval}{%
  Vom Stuttgarter Zentrum für Simulationswissenschaften (SC SimTech) und\\
  der Fakultät für Informatik, Elektrotechnik und Informationstechnik\\
  der Universität Stuttgart zur Erlangung der Würde eines Doktors\\
  der Naturwissenschaften (Dr.\ rer.\ nat.) genehmigte Abhandlung%
}
\newcommand*{\theauthor}{Benjamin Maier}
\newcommand*{\thebirthplace}{Waiblingen}
\newcommand*{\thedefensedate}{22. Juni 2021}
\newcommand*{\thedate}{April 22, 2021}
\newcommand*{\theyear}{2021}
\newcommand*{\theuniversity}{Universität Stuttgart}
\newcommand*{\theinstitute}{Institut für Parallele und Verteilte Systeme}
\newcommand*{\theadvisor}{Prof.\ Dr.\ Miriam Schulte}
\newcommand*{\theexamineri}{Prof.\ Dr.\ Hans-Joachim Bungartz}
\newcommand*{\theexaminerii}{}

% ======================================================================
% KOMA-Script and Text Area
% ======================================================================

% options for KOMA-Script
\KOMAoptions{
  % choose font size (PO: should be between 12pt and 14pt)
  fontsize=12pt,
  % suppress "very small head height detected" warning
  DIV=12,
  % don't end section numbers with period despite appendices
  numbers=noendperiod,
  % set linespacing of header and footer to 1.5
  % (otherwise, the header will "jump" back and forth from normal
  % pages and pages with single line spacing, e.g., table of contents)
  onpsinit=\onehalfspacing,
  % use \section for headings of list of figures/tables/algorithms
  listof=leveldown,
}

% binding offset that will be substracted from inner margins
\newcommand*{\bindingoffset}{10mm}

% set page margins
\geometry{
  bindingoffset=\bindingoffset,
  inner=15mm,
  outer=30mm,
  top=20mm,
  bottom=30mm,
  includehead=true,
}

% ======================================================================
% Graphics
% ======================================================================

% location of graphics files
\graphicspath{{../gfx/}}

\graphicspath{
{images/summer_school_study/png/}
{images/summer_school_study/}
{images/summer_school_study/plots/}
{images/summer_school_study/2018/}
{images/fiber_creation/}
{images/parallel_fiber_estimation/}
{images/motor_unit_assignment/}
{images/implementation/}
{images/theory/}
{images/results/basic}
{images/results/application}
{images/results/studies}
{images/logos}
}

% TikZ libraries
\usetikzlibrary{
  % arrow types
  arrows.meta,
  % coordinate calculations
  calc,
  % zig-zag lines
  decorations.pathmorphing,
  % curly braces
  decorations.pathreplacing,
  % text along paths
  decorations.text,
  % color gradients
  fadings,
  % shadows
  shadows,
}

% set default line width to same as in Matplotlib plots
% created by helper.figure
\tikzset{every picture/.style={line width=1pt}}

% set default arrow tip
\tikzset{>={Stealth[length=5.5pt,width=5.5pt]}}

% draw circle arc around center (standard in TikZ is
% the specification of the first point on the arc)
% [draw options] (center) (initial angle:final angle:radius)
\def\centerarc[#1](#2)(#3:#4:#5){
  \draw[#1] ($(#2)+({(#5)*cos(#3)},{(#5)*sin(#3)})$) arc (#3:#4:#5);
}

% width of contours around text with \contour
\contourlength{1.5pt}

% ======================================================================
% Cross-References
% ======================================================================

% define abbreviated names for cross-references
\crefname{algorithm}{Alg.}{Algorithms}
\Crefname{algorithm}{Algorithm}{Algorithms}

\crefname{chapter}{Chap.}{Chapters}
\Crefname{chapter}{Chapter}{Chapters}

\crefname{corollary}{Cor.}{Corollaries}
\Crefname{corollary}{Corollary}{Corollaries}

\crefname{definition}{Def.}{Definitions}
\Crefname{definition}{Definition}{Definitions}

\crefname{equation}{Eq.}{Equations}
\Crefname{equation}{Equation}{Equations}

\crefname{figure}{Fig.}{Figures}
\Crefname{figure}{Figure}{Figures}

\crefname{lemma}{Lemma}{Lemmas}
\Crefname{lemma}{Lemma}{Lemmas}

\crefname{line}{line}{lines}
\Crefname{line}{Line}{Lines}

\crefname{proposition}{Prop.}{Propositions}
\Crefname{proposition}{Proposition}{Propositions}

\crefname{section}{Sec.}{Sections}
\Crefname{section}{Section}{Sections}

\crefname{table}{Tab.}{Tables}
\Crefname{table}{Table}{Tables}

\crefname{theorem}{Thm.}{Theorems}
\Crefname{theorem}{Theorem}{Theorems}

% reference theorems with their name appended
\newcommand*{\thmref}[1]{\hyperlink{#1}{\cref*{#1}~(\nameref*{#1})}}
\newcommand*{\Thmref}[1]{\hyperlink{#1}{\Cref*{#1}~(\nameref*{#1})}}

% ======================================================================
% Tables of Contents, Figures, Tables, Algorithms, and Theorems
% ======================================================================

% remove indentation for lists of figures and tables
\DeclareTOCStyleEntry[indent=0em]{tocline}{figure}
\DeclareTOCStyleEntry[indent=0em]{tocline}{table}

% -- this fails
% declare algorithm float environment, create list of algorithms,
% use same formatting as for figures and tables
%\DeclareNewTOC[
%  type=algorithm,
%  name=Algorithm,
%  listname={List of Algorithms},
%  float,
%  floattype=4,
%  floatpos=tp,
%  counterwithin=chapter,
%  tocentrynumwidth=2.3em,
%  tocentryindent=0em,
%]{loa}

% create list of theorems,
% use same formatting as for figures, tables, and algorithms
% (thmtools's thm-listof.sty is unfortunately a little buggy,
% as it causes the flip book images to jump; using KOMA-Script for
% all lists is more consistent and less error-prone)
\DeclareNewTOC[
  type=mytheorem,
  name=Theorem,
  listname={List of Theorems},
  tocentrynumwidth=2.3em,
  tocentryindent=0em,
]{lop}

% add code that adds the theorem to the *.lop file to hook
\addtotheorempostheadhook{%
  \addxcontentsline{lop}{mytheorem}{\csname ll@\thmt@envname\endcsname}%
}

% use single spacing in
% tables of contents, figures, tables, algorithms, and theorems
\AfterTOCHead{\begin{spacing}{1}}
\AfterStartingTOC{\end{spacing}}

% exclude subsections from table of contents
\setcounter{tocdepth}{\sectionnumdepth}

% increase space of page numbers
% (default values of 1.55em and 2.55em are too small for three-digit numbers)
\renewcommand*{\@pnumwidth}{2em}
\renewcommand*{\@tocrmarg}{3em}

% ======================================================================
% Dictums
% ======================================================================

% don't use sans-serif font for dictums
\addtokomafont{dictum}{\rmfamily}

% shortcut for setting the dictum for a chapter
\newcommand*{\setdictum}[3][0.54\textwidth]{%
  \setchapterpreamble{%
    \cleanchapterquote{#1}{#2}{#3}%
    \chapterheadendvskip%
  }%
}

% fancy quotes (taken from cleanthesis.sty, slightly adapted)
\newcommand*{\hugequote}{%
  \fontsize{75}{80}\selectfont%
  \hspace*{-0.6em}\color[rgb]{0.6,0.6,0.6}%
  \textit{``}%
  \vskip -0.8em%
}

\newcommand*{\cleanchapterquote}[3]{%
  \begin{minipage}{\textwidth}%
    \begin{flushright}
      \begin{minipage}{#1}%
        \begin{flushleft}
          {\hugequote}\textit{#2}
        \end{flushleft}
        \begin{flushright}
          \small--- #3
        \end{flushright}
      \end{minipage}%
    \end{flushright}
  \end{minipage}%
  \bigskip
}

% ======================================================================
% Page Headers and Footers
% ======================================================================

% all-caps sans-serif page header
\setkomafont{pageheadfoot}{%
  \normalfont\normalcolor\sffamily%
  \fontsize{9.5}{12}\selectfont\lsstyle%
}
\setkomafont{pagenumber}{\usekomafont{pageheadfoot}\bfseries}

% move page number to page header
\clearscrheadfoot
\lehead[\pagemark]{\pagemark}
\rehead[]{\headmark}
\lohead[]{\headmark}
\rohead[\pagemark]{\pagemark}
\lefoot[]{}
\rofoot[]{}

% prepend "Chapter X: " in front of the chapter name in page headers
\renewcommand*{\chaptermarkformat}{\chapapp{} \thechapter: }

% ======================================================================
% Initials
% ======================================================================

% indent right of initial
\setlength{\DefaultNindent}{0.5em}
% use height of upper-case letters for height of initial
\renewcommand*{\LettrineSecondString}{X}
% enlarge initial (grows to the top, aligned on baseline of bottom line)
\renewcommand*{\DefaultLoversize}{0.07}
% make initial bold
\renewcommand*{\LettrineFontHook}{\bfseries}
% don't use small caps for rest of first word, make bold
\renewcommand*{\LettrineTextFont}{\normalfont\bfseries}

% custom command: \initial[options]{0.5em}{L}{etter}
% is like \lettrine[options]{L}{etter}, except that the first line
% is moved to the left by 0.5em
\newcommand*{\initial}[4][]{%
  \lettrine[%
    findent=\dimexpr-#2\relax,%
    nindent=\dimexpr#2+\DefaultNindent\relax,%
    #1%
  ]{#3}{#4}%
}

% ======================================================================
% Tables
% ======================================================================

% fixed-width column types with left, center, and right alignment
\newcolumntype{L}[1]{%
  >{\raggedright\let\newline\\\arraybackslash\hspace{0pt}}p{#1}%
}
\newcolumntype{D}[1]{%
  >{\centering\let\newline\\\arraybackslash\hspace{0pt}}p{#1}%
}
\newcolumntype{R}[1]{%
  >{\raggedleft\let\newline\\\arraybackslash\hspace{0pt}}p{#1}%
}

% colors of table lines and rows
\newcommand*{\tablelinecolor}{mittelblau}
\newcommand*{\headerrowcolor}{mittelblau!40}
\newcommand*{\headerrowtextcolor}{black}
\newcommand*{\oddrowcolor}{mittelblau!10}
\newcommand*{\evenrowcolor}{mittelblau!20}

% \setnumberoftableheaderrows has to be put before every tabular environment,
% sets the \rownum counter such that the 1st content row has row number 1
\newcommand*{\setnumberoftableheaderrows}[1]{%
  \rowcolors{1}{\oddrowcolor}{\evenrowcolor}%
  \global\rownum=\numexpr-#1\relax%
}

% set text color per row, use it like this: \begin{tabular}{=l+l+l+l} ...
% (a single \color command wouldn't suffice, as tabular cells have
% their own boxes, i.e., the font color is reset after every cell)
\newcommand*{\@rowstyle}{}
\newcommand*{\rowstyle}[1]{\gdef\@rowstyle{#1}\@rowstyle\ignorespaces}
\newcolumntype{=}{>{\gdef\@rowstyle{}}C}
\newcolumntype{+}{>{\@rowstyle}C}

% \headerrow has to be put before every tabular header row,
% sets background and text color
\newcommand*{\headerrow}{%
  \rowcolor{\headerrowcolor}%
  \rowstyle{\color{\headerrowtextcolor}}%
}

% automatically select header or odd/even row color
% for the current tabular row
\newcommand*{\autorowcolor}{%
  \ifnum\rownum<1%
    \headerrowcolor%
  \else%
    \ifodd\rownum\oddrowcolor\else\evenrowcolor\fi%
  \fi%
}

% automatically select header or odd/even row color
% for the previous tabular row
\newcommand*{\prevautorowcolor}{%
  \ifnum\rownum<2%
    \headerrowcolor%
  \else%
    \ifodd\rownum\evenrowcolor\else\oddrowcolor\fi%
  \fi%
}

% set thickness of table rules
\setlength{\heavyrulewidth}{0.10em}
\setlength{\lightrulewidth}{0.06em}

% like \toprule, except that the space below the rule has the color
% given in the argument
\newcommand*{\toprulecustom}[1]{%
  \arrayrulecolor{\tablelinecolor}%
  \specialrule{\heavyrulewidth}{\abovetopsep}{0pt}%
  \arrayrulecolor{#1}%
  \specialrule{\belowrulesep}{0pt}{0pt}%
  \arrayrulecolor{black}%
}

% like \midrule, except that the space above and below the rule has the colors
% given in the arguments
\newcommand*{\midrulecustom}[2]{%
  \arrayrulecolor{#1}%
  \specialrule{\aboverulesep}{0pt}{0pt}%
  \arrayrulecolor{\tablelinecolor}%
  \specialrule{\lightrulewidth}{0pt}{0pt}%
  \arrayrulecolor{#2}%
  \specialrule{\belowrulesep}{0pt}{0pt}%
  \arrayrulecolor{black}%
}

% like \bottomrule, except that the space above the rule has the color
% given in the argument
\newcommand*{\bottomrulecustom}[1]{%
  \arrayrulecolor{#1}%
  \specialrule{\aboverulesep}{0pt}{0pt}%
  \arrayrulecolor{\tablelinecolor}%
  \specialrule{\heavyrulewidth}{\belowbottomsep}{0pt}%
  \arrayrulecolor{black}%
}

% commands like \toprule, \midrule, and \bottomrule,
% but automatically guessing the right background colors
% for the spaces above/below the rules
\newcommand*{\toprulec}{\toprulecustom{\autorowcolor}}
\newcommand*{\midrulec}{%
  \midrulecustom{\prevautorowcolor}{\autorowcolor}%
}
\newcommand*{\bottomrulec}{\bottomrulecustom{\prevautorowcolor}}

% ======================================================================
% Algorithms
% ======================================================================

% disable ligatures ("fi" etc.) in monospace font
\DisableLigatures{encoding=*,family=tt*}

% line number format
\algrenewcommand{\alglinenumber}[1]{\footnotesize\color{anthrazit}{\texttt{#1}}}
\algrenewcommand{\algorithmicrequire}{\qquad\textbf{Input:}}
\algrenewcommand{\algorithmicensure}{\qquad\textbf{Output:}}

% function name
\algrenewcommand{\textproc}{}

% bold keywords
\algnewcommand{\Break}{\textbf{break}}
\algnewcommand{\Continue}{\textbf{continue}}
\algnewcommand{\True}{\textbf{true}}
\algnewcommand{\False}{\textbf{false}}
\algnewcommand{\Null}{\textbf{null}}
\algrenewcommand{\algorithmicend}{\textbf{end}}
\algrenewcommand{\algorithmicdo}{\textbf{do}}
\algrenewcommand{\algorithmicwhile}{\textbf{while}}
\algrenewcommand{\algorithmicfor}{\textbf{for}}
\algrenewcommand{\algorithmicforall}{\textbf{for all}}
\algrenewcommand{\algorithmicloop}{\textbf{loop}}
\algrenewcommand{\algorithmicrepeat}{\textbf{repeat}}
\algrenewcommand{\algorithmicuntil}{\textbf{until}}
\algrenewcommand{\algorithmicprocedure}{\textbf{procedure}}
\algrenewcommand{\algorithmicfunction}{\textbf{function}}
\algrenewcommand{\algorithmicif}{\textbf{if}}
\algrenewcommand{\algorithmicthen}{\textbf{then}}
\algrenewcommand{\algorithmicelse}{\textbf{else}}
\algrenewcommand{\algorithmicreturn}{\textbf{return}}
\algnewcommand{\algorithmicforever}{\textbf{for ever}}
\algdef{S}[FOR]{ForEver}{\algorithmicforever\ \algorithmicdo}
\algnewcommand{\algorithmicgoto}{\textbf{go to}}
\algnewcommand{\Goto}[1]{\algorithmicgoto\ \cref*{#1}}
\algnewcommand{\ForOneLine}[2]{%
  \State\algorithmicfor\ #1\ \algorithmicdo\ #2%
}
\algnewcommand{\IfOneLine}[2]{%
  \State\algorithmicif\ #1\ \algorithmicthen\ #2%
}
\algnewcommand{\ElseOneLine}[1]{%
  \State\algorithmicelse\ #1%
}

% small monospace font in algorithms
\g@addto@macro\ALG@beginalgorithmic{\small\ttfamily}

% comment format
\algrenewcomment[1]{\hfill$\rightsquigarrow$\;{\normalfont\emph{#1}}}

% set indentation to two characters
\algrenewcommand{\algorithmicindent}{\widthof{AB}}

% make ALG@line.XX counters unique,
% otherwise hyperlinks to algorithm lines won't work
% (they will always point to the line with the same number,
% but in the first algorithm of the thesis)
\newcounter{algorithmicH}
\let\oldalgorithmic\algorithmic
\renewcommand*{\algorithmic}{\stepcounter{algorithmicH}\oldalgorithmic}
\renewcommand*{\theHALG@line}{ALG@line.\thealgorithmicH.\arabic{ALG@line}}

% ======================================================================
% Mathematics
% ======================================================================

% automatically replace "l" with \ell in math mode
\mathcode`l="8000
\begingroup
\lccode`\~=`\l
\DeclareMathSymbol{\lsb@l}{\mathalpha}{letters}{`l}
\lowercase{\gdef~{\ifnum\the\mathgroup=\m@ne \ell \else \lsb@l \fi}}%
\endgroup

% change QED symbol to filled square
%\renewcommand*{\qedsymbol}{\textcolor{mittelblau}{\blacksquare}}

% redefine own proof environment (prohibits \qedhere)
%\renewenvironment{proof}[1][Proof]{%
%  \noindent{\formatcaption{#1}\hspace{1em}}%
%}{\qed\par}

% siunitx setup
\sisetup{
  % reduce space between units
  inter-unit-product={\hspace{0.03em}},
  % use fractions for inverse units
  per-mode=fraction,
  % use dot for product between number and power of ten
  exponent-product=\cdot,
}

% add \permille macro
\DeclareSIUnit{\permille}{\text{\textperthousand}}

% add \hms command similar to \ang to describe durations
% (e.g., computation times), use as \hms{1;2;3} ==> 1h 02min 03s
\ExplSyntaxOn
\NewDocumentCommand\hms{o>{\SplitArgument{2}{;}}m}{
  \group_begin:
  \IfNoValueF{#1}{\keys_set:nn{siunitx}{#1}}
  \siunitx_hms_output:nnn #2
  \group_end:
}
\cs_new_protected:Npn \siunitx_hms_output:nnn #1#2#3
{
  \IfNoValueF{#1}{
    \tl_if_blank:nF{#1}{
      \SI{#1}{\hour}
      \IfNoValueF{#2}{~}
    }
  }
  \IfNoValueF{#2}{
    \tl_if_blank:nF{#2}{
      \IfNoValueF{#1}{\tl_if_blank:nF{#1}{\sisetup{minimum-integer-digits=2}}}
      \SI{#2}{\minute}
      \IfNoValueF{#3}{~}
    }
  }
  \IfNoValueF{#3}{
    \tl_if_blank:nF{#3}{
      \IfNoValueF{#1}{\tl_if_blank:nF{#1}{\sisetup{minimum-integer-digits=2}}}
      \IfNoValueF{#2}{\tl_if_blank:nF{#2}{\sisetup{minimum-integer-digits=2}}}
      \SI{#3}{\second}
    }
  }
}
\ExplSyntaxOff

% make spacing after partial sign smaller
\edef\partial{\mathchar\number\partial\noexpand\mkern-2mu}

% ======================================================================
% Dummy Text
% ======================================================================

% insert dummy text with TODO, warn for every usage
\newcommand*{\dummytext}[1][\value{blindtext}]{%
  \todo{write}
  \textcolor{gray}{\blindtext[#1]}%
}

% ======================================================================
% Dynamic Commands
% ======================================================================

% get name of current label (chapter, section, subsection, ...)
\newcommand*{\currentname}{\@currentlabelname}

% calculate difference between page numbers
\newcommand*{\pagedifference}[2]{%
  \number\numexpr\getpagerefnumber{#2}-\getpagerefnumber{#1}\relax%
}

% custom TODO command with warning
% (all packages that were tried produced problems)
\newcommand*{\todo}[1]{%
  \GenericWarning{}{%
    LaTeX Warning (\thesubsection\space\currentname): TODO "#1"%
  }%
  \textcolor{red}{TODO: #1}%
}

% set up versioning information
%\luaexec{require("version")}

% convenience commands for title page and watermarks
\newcommand*{\gitCommitText}[1][]{%
  \texttt{#1\gitCommitHash}%
  \ifx\gitTag\empty\else{} (\texttt{#1\gitTag})\fi%
}

\newcommand*{\compileCounterText}[1][]{%
  \texttt{#1v\compileCounter}%
}

% command for checking if file is in \includeonly or not
\newcommand*{\isincluded}[1]{%
  \@tempswatrue
  \if@partsw
    \@tempswafalse
    \edef\reserved@b{#1}%
    \@for\reserved@a:=\@partlist\do
    {\ifx\reserved@a\reserved@b\@tempswatrue\fi}%
  \fi
  \if@tempswa\expandafter\@firstoftwo\else\expandafter\@secondoftwo\fi
}

% ======================================================================
% Text Body
% ======================================================================

% increase indentation of paragraphs (default: 1em = \quad)
\setlength{\parindent}{1em}

% prohibit inserting page breaks in footnotes
\interfootnotelinepenalty=10000

% only display underfull \hbox/\vbox warnings if really, really bad;
% in a technical text riddled with long inline formulas and narrow
% sub-captions or side-captions, it's just not possible to avoid all
% badnesses over 1000
\hbadness=9999
\vbadness=9999

% mathdesign's Charter font has some questionable kerning between some
% capital letters such as F, P, and V and a period (the space between those
% is far too small); one solution would be to use the charter package for
% the main text, but that leads a number of other issues (the text size is
% then larger, and doesn't match the math font size anymore...)
\newcommand*{\punctfix}[1]{{\hspace{-0.05em}#1}}

% enumerate and itemize spaces
\setlist{
  % space between left edge of text area and left edge of paragraphs
  leftmargin=\parindent,
  % space between item paragraphs (in addition to \parskip)
  itemsep=0pt,
  % space above and below lists
  topsep=0.5em,
}

% math version of \settowidth, automatically choosing the right style
% (textstyle, displaystyle, ...)
\def\mathsettowidth#1#2{%
  \setbox\@tempboxa\hbox{$\m@th\mathpalette{}{#2}$}%
  #1=\wd\@tempboxa%
  \setbox\@tempboxa\box\voidb@x%
}

% \halfhphantom works like \hphantom, except that it creates a box
% that is only half as wide as that of \hphantom
\newlength{\halfhphantomlength}
\newcommand*{\halfhphantom}[1]{%
  \ifmmode%
    \mathsettowidth{\halfhphantomlength}{#1}%
  \else%
    \settowidth{\halfhphantomlength}{#1}%
  \fi%
  \setlength{\halfhphantomlength}{\halfhphantomlength/2}%
  \hspace{\halfhphantomlength}%
}

% \lefthphantom{abc}{defghij} positions the text "abc" as follows:
% |abc    |
%  defghij
\newcommand*{\lefthphantom}[2]{%
  \ifmmode\mathrlap{#1}\else\rlap{#1}\fi%
  \hphantom{#2}%
}

% \centerhphantom{abc}{defghij} positions the text "abc" as follows:
% |  abc  |
%  defghij
\newcommand*{\centerhphantom}[2]{%
  \halfhphantom{#2}%
  \ifmmode\mathclap{#1}\else\clap{#1}\fi%
  \halfhphantom{#2}%
}

% \righthphantom{abc}{defghij} positions the text "abc" as follows:
% |    abc|
%  defghij
\newcommand*{\righthphantom}[2]{%
  \hphantom{#2}%
  \ifmmode\mathllap{#1}\else\llap{#1}\fi%
}

% use Oxford comma for multiple references in cleveref
% (e.g, \cref{fig1,fig2,fig3} => Fig. 1, 2, and 3)
\newcommand*{\creflastconjunction}{, and\nobreakspace}

% ======================================================================
% Colors
% ======================================================================

% define line colors (mix between MATLAB and matplotlib colors)
\definecolor{C0}{rgb}{0.000,0.447,0.741}
\definecolor{C1}{rgb}{0.850,0.325,0.098}
\definecolor{C2}{rgb}{0.749,0.561,0.102}
\definecolor{C3}{rgb}{0.494,0.184,0.556}
\definecolor{C4}{rgb}{0.466,0.674,0.188}
\definecolor{C5}{rgb}{0.301,0.745,0.933}
\definecolor{C6}{rgb}{0.635,0.078,0.184}
\definecolor{C7}{rgb}{0.887,0.465,0.758}
\definecolor{C8}{rgb}{0.496,0.496,0.496}

% define university CD colors
\definecolor{anthrazit}{RGB}{62,68,76}
\definecolor{mittelblau}{RGB}{0,81,158}
\definecolor{hellblau}{RGB}{0,190,255}

% ======================================================================
% PDF Meta-Data and Links
% ======================================================================

% set up hyperref
\hypersetup{
  % set metadata
  pdftitle={\thetitle},
  pdfauthor={\theauthor},
  pdfcreator={LaTeX, KOMA-Script, hyperref},
  % underline links instead of putting a framed box around them
  pdfborderstyle={/S/U/W 1},
  % set link colors
  citebordercolor=C1,
  filebordercolor=C1,
  linkbordercolor=C1,
  menubordercolor=C1,
  runbordercolor=C1,
  urlbordercolor=C0,
  % prepend bookmarks with section number
  bookmarksnumbered,
  % open bookmark tree on start
  bookmarksopen,
}

% include parentheses of \eqref in hyperlink
\renewcommand*{\eqref}[1]{\hyperref[{#1}]{\textup{\tagform@{\ref*{#1}}}}}

% ======================================================================
% Bibliography
% ======================================================================

% location of *.bib file
\addbibresource{document/references.bib}

% include brackets in \cite hyperlink
%\DeclareCiteCommand{\cite}{\usebibmacro{prenote}}{%
%  \usebibmacro{citeindex}%
 % \printtext[bibhyperref]{\mkbibbrackets{\usebibmacro{cite}}}%
%}{\multicitedelim}{\usebibmacro{postnote}}

% declare special \multicite command which produces [Abc01; Def02]
% (instead of [Abc01]; [Def02])
\DeclareCiteCommand{\cite}[\mkbibbrackets]{\usebibmacro{prenote}}{%
  \usebibmacro{citeindex}\usebibmacro{cite}%
}{\multicitedelim}{\usebibmacro{postnote}}

% don't append "+" sign to BibLaTeX citations in alphabetic style
\renewcommand*{\labelalphaothers}{}

% define custom heading to add bibliography to table of contents
\defbibheading{myheading}[\bibname]{\addchap{#1}}

% add custom bibliography post note (reset font size)
\defbibnote{mypostnote}{%
  \renewcommand*{\texttt}[1]{\oldtexttt{##1}}%
}

% order entries by last names
\DeclareNameAlias{sortname}{last-first}
\DeclareNameAlias{default}{last-first}

% change font size of bibliography
\renewcommand*{\bibfont}{\small}

% set single line spacing for bibliography
\renewcommand*{\bibsetup}{\singlespacing}

% separate authors with semicolon, suppress "and" in author names
\renewcommand*{\multinamedelim}{\addsemicolon\space}
\renewcommand*{\finalnamedelim}{\addsemicolon\space}

% make names bold
\renewcommand*{\mkbibnamefamily}[1]{\textbf{#1}}

% insert colon after author names in bibliography
\renewcommand*{\labelnamepunct}{\addcolon\space}

% separate BibLaTeX units with commas instead of periods
\renewcommand*{\newunitpunct}{\addcomma\space}

% make paper titles italic, remove quotation marks
\DeclareFieldFormat[article]{title}{\mkbibemph{#1}}
\DeclareFieldFormat[article]{journaltitle}{#1}
\DeclareFieldFormat[article]{issuetitle}{#1}
\DeclareFieldFormat[article]{issuesubtitle}{#1}
\DeclareFieldFormat[inbook]{title}{\mkbibemph{#1}}
\DeclareFieldFormat[inbook]{booktitle}{#1}
\DeclareFieldFormat[inbook]{maintitle}{#1}
\DeclareFieldFormat[incollection]{title}{\mkbibemph{#1}}
\DeclareFieldFormat[incollection]{booktitle}{#1}
\DeclareFieldFormat[incollection]{maintitle}{#1}
\DeclareFieldFormat[inproceedings]{title}{\mkbibemph{#1}}
\DeclareFieldFormat[inproceedings]{booktitle}{#1}
\DeclareFieldFormat[inproceedings]{maintitle}{#1}
\DeclareFieldFormat[inreference]{booktitle}{#1}
\DeclareFieldFormat[inreference]{maintitle}{#1}
\DeclareFieldFormat[thesis]{title}{\mkbibemph{#1}}
\DeclareFieldFormat[unpublished]{title}{\mkbibemph{#1}}

% fix title capitalization to sentence case (all lowercase)
\DeclareFieldFormat{titlecase}{\MakeTitleCase{#1}}

% .. but don't change journal titles, book titles, and so on
\newrobustcmd{\MakeTitleCase}[1]{%
  \ifthenelse{%
    \ifcurrentfield{booktitle}\OR\ifcurrentfield{booksubtitle}%
    \OR\ifcurrentfield{maintitle}\OR\ifcurrentfield{mainsubtitle}%
    \OR\ifcurrentfield{journaltitle}\OR\ifcurrentfield{journalsubtitle}%
    \OR\ifcurrentfield{issuetitle}\OR\ifcurrentfield{issuesubtitle}%
    \OR\ifentrytype{book}\OR\ifentrytype{mvbook}\OR\ifentrytype{bookinbook}%
    \OR\ifentrytype{booklet}\OR\ifentrytype{suppbook}%
    \OR\ifentrytype{collection}\OR\ifentrytype{mvcollection}%
    \OR\ifentrytype{suppcollection}\OR\ifentrytype{manual}%
    \OR\ifentrytype{periodical}\OR\ifentrytype{suppperiodical}%
    \OR\ifentrytype{proceedings}\OR\ifentrytype{mvproceedings}%
    \OR\ifentrytype{reference}\OR\ifentrytype{mvreference}%
    \OR\ifentrytype{thesis}\OR\ifentrytype{online}%
  }{%
    #1%
  }{%
    \MakeSentenceCase*{#1}%
  }%
}

% suppress "In:" before journal names
\renewbibmacro*{in:}{}

% don't put parentheses around year for journals
\renewbibmacro*{issue+date}{%
  \setunit{\addcomma\space}%
  \iffieldundef{issue}{%
    \usebibmacro{date}%
  }{%
    \printfield{issue}\setunit*{\addspace}\usebibmacro{date}%
  }%
  \newunit%
}

% reformat and link URLs, DOIs, and ISBNs
\DeclareFieldFormat{url}{\scriptsize\url{#1}}
\DeclareFieldFormat{doi}{%
  \scriptsize%
  \ifhyperref{%
    \href{https://doi.org/#1}{\nolinkurl{doi:#1}}%
  }{\nolinkurl{doi:#1}}%
}
\DeclareFieldFormat{isbn}{%
  \scriptsize%
  \ifhyperref{%
    \href{https://www.amazon.com/s/?field-keywords=#1}{\nolinkurl{isbn:#1}}%
  }{\nolinkurl{isbn:#1}}%
}

% hide urldate field
\AtEveryBibitem{\clearfield{urlyear}}

% remove period at end of entries
\renewcommand*{\finentrypunct}{}

\DeclareBibliographyCategory{own}
\DeclareBibliographyCategory{own_other}
\DeclareBibliographyCategory{own_forthcoming}


\makeatother

    

%%%%%%%%%%%%%%%%   Abkürzungen   %%%%%%%%%%%%%%%%%%

%----------------------Umgebungen----------------------
\def\beqno{\begin{equation}}
\def\eeqno{\end{equation}}
\def\beq{\begin{equation*}}
\def\eeq{\end{equation*}}
\def\ba#1{\begin{array}{#1}}
\def\ea{\end{array}}
\def\mat#1{\left(\begin{matrix}#1\end{matrix}\right)}
\def\matt#1{\left[\begin{matrix}#1\end{matrix}\right]}
%\newcommand{\bs}{\begin{subequations}}
%\newcommand{\es}{\end{subequations}}
\newcommand{\bal}{\begin{align}}
\def\ealll{ \end{align} }

\renewcommand{\emph}[1]{\textit{#1}\/}
\def\clap#1{\hbox  to  0pt{\hss#1\hss}}                 % für underbrace
\def\mathclap{\mathpalette\mathclapinternal}
\def\mathclapinternal#1#2{\clap{$\mathsurround=0pt#1{#2}$}}
\newcommand{\ub}[2]{\underbrace{#1}_{\mathclap{#2}}}    
\newcommand{\ds}{\displaystyle}                         % displaystyle
\renewcommand{\dfrac}[2]{\ds\frac{\ds{#1}}{\ds{#2}}\,}  % nach Bruch Abstand
%\newcommand{\code}[1]{{\small\lstinline[columns=fixed]!#1!}}

\usepackage{setspace}
\newcommand{\code}[1]{{\small\lstinline[basicstyle=\small\ttfamily\color{Maroon},breaklines=true,upquote=true]!#1!}}
\newcommand{\codenobreak}[1]{{\small\lstinline[basicstyle=\small\ttfamily\color{Maroon},breaklines=false,upquote=true]!#1!}}
\newcommand{\emphcode}[1]{{\small\lstinline[basicstyle=\small\itshape\ttfamily\color{Maroon},breaklines=true]!#1!}}
\newcommand{\codebox}[1]{\begin{lstlisting}[columns=fullflexible,breaklines=true,postbreak=\mbox{\textcolor{gray}{$\hookrightarrow$}\space}]#1\end{lstlisting}}

%\newfloat{algorithm}{ht}{aux0}              % Algorithmus-Umgebung
%\floatname{algorithm}{Code-Abschnitt}
\newcommand{\anm}[1]{\textcolor{blue}{#1}}
\def\bigA{\mathop{\mathrm{A}}}

% -------------
% Using mdframed, define environment for \begin{reproduce} abcdefgh \end{reproduce}
\definecolor{Maroon}{HTML}{801010}
%\mdfsetup{skipabove=\topskip,skipbelow=\topskip}
\newcounter{reproduce}[section]
\newenvironment{reproduce}{%
  \stepcounter{reproduce}%
  \mdfsetup{%
    frametitle={%
      \tikz[baseline=(current bounding box.east),outer sep=0pt]
      \node[anchor=east,rectangle,fill=red!20]
      {How To Reproduce};},nobreak=false
  }
  \mdfsetup{innertopmargin=10pt,linecolor=Maroon!20,nobreak=false,backgroundcolor=Maroon!05,%
            linewidth=2pt,topline=true,roundcorner=5pt,frametitleaboveskip=\dimexpr-\ht\strutbox\relax,%
  }
  \vfill
  %\begin{minipage}[b]{\linewidth}
  \begin{mdframed}\relax%
}
{
  \end{mdframed}%
  %\end{minipage}%
}

\newenvironment{reproduce_no_break}{%
  \stepcounter{reproduce}%
  \mdfsetup{%
    frametitle={%
      \tikz[baseline=(current bounding box.east),outer sep=0pt]
      \node[anchor=east,rectangle,fill=red!20]
      {How To Reproduce};},nobreak=true
  }
  \mdfsetup{innertopmargin=10pt,linecolor=Maroon!20,nobreak=true,backgroundcolor=Maroon!05,%
            linewidth=2pt,topline=true,roundcorner=5pt,frametitleaboveskip=\dimexpr-\ht\strutbox\relax,%
  }
  \vfill
  %\begin{minipage}[b]{\linewidth}
  \begin{mdframed}\relax%
}
{
  \end{mdframed}%
  %\end{minipage}%
}

% -------------


%----------------------Funktionen, Zeichen----------------------
\def\det{\hbox{det} \,}
\def\spn{\hbox{span} \,}
\def\div{\hbox{div} \,}
\def\grad{\hbox{grad} \,}
\def\supp{\hbox{supp} \,}
\def\cof{\hbox{cof} \,}
\def\tr{\hbox{tr} \,}
\def\sym{\hbox{sym} \,}
%\def\spur{\hbox{\textup{spur}} \,}
\DeclareMathOperator{\spur}{spur}
\newcommand{\stern}[1] {\overset{*}{#1}}        %Sternchen auf Buchstabe
\def\tstern{\stern{t}}
\def\dV{\d V}
\def\qed{\begin{flushright}$\square$\end{flushright}}
%\renewcommand{\grqq}{\grqq\,}
\def\rpsi{\textcolor{red}{\hat{\psi}}}
\def\Dcon{\mathcal{D}_{con}}
\def\Dloc{\mathcal{D}_{loc}}
\def\D{\mathcal{D}}
\def\E{\mathbb{E}} % C2 domain of elasticity
\def\P{\mathcal{P}} % C2 domain of elasticity
\def\dt{d\noexpand\mkern-3mu t}
\DeclareSIUnit[number-unit-product = {}]{\flop}{FLOP}   % flops unit for unitx
\DeclareSIUnit[number-unit-product = {}]{\flops}{FLOPS}   % flops unit for unitx
\sisetup{retain-unity-mantissa = false}

%----------------------Ableitungen------------------------

%Ableitungen mit \d 
\makeatletter
\def\d{\futurelet\next\start@i}\def\start@i{\ifx\next\bgroup\expandafter\abl@\else\expandafter\abl@d\fi}\def\abl@#1{\def\tempa{#1}\futurelet\next\abl@i}\def\abl@i{\ifx\next\bgroup\expandafter\abl@ii\else\expandafter\abl@a\fi}\def\abl@ii#1{\def\tempb{#1}\futurelet\next\abl@iii}\def\abl@iii{\ifx\next\bgroup\expandafter\abl@c\else\expandafter\abl@b\fi}
\def\abl@d{\mathrm{d}}                                          % keine Argumente
\def\abl@a{\ds\frac{\mathrm{d}}{\mathrm{d}\tempa}\,}            % 1 Argument \d{x} -> d/dx
\def\abl@b{\ds\frac{\mathrm{d}\tempa}{\mathrm{d}\tempb}\,}  % 2 Argumente: \d{f}{x} -> df/dx
\def\abl@c#1{\ds\frac{\mathrm{d}^{#1} {\tempa}}{\mathrm{d} {\tempb}^{#1}}\,}        % 3 Argumente: \d{f}{x}{2} -> d^2f/dx^2

%partielle Ableitungen mit \p
\def\p{\futurelet\next\startp@i}\def\startp@i{\ifx\next\bgroup\expandafter\pabl@\else\expandafter\pabl@d\fi}\def\pabl@#1{\def\tempa{#1}\futurelet\next\pabl@i}\def\pabl@i{\ifx\next\bgroup\expandafter\pabl@ii\else\expandafter\pabl@a\fi}\def\pabl@ii#1{\def\tempb{#1}\futurelet\next\pabl@iii}\def\pabl@iii{\ifx\next\bgroup\expandafter\pabl@c\else\expandafter\pabl@b\fi}
\def\pabl@d{\partial}                                           % keine Argumente
\def\pabl@a{\ds\frac{\partial}{\partial\tempa}\,}           % 1 Argument \d{x} -> d/dx
\def\pabl@b{\ds\frac{\partial\tempa}{\partial\tempb}\,} % 2 Argumente: \d{f}{x} -> df/dx
\def\pabl@c#1{\ds\frac{\partial^{#1} {\tempa}}{\partial {\tempb}^{#1}}\,}       % 3 Argumente: \d{f}{x}{2} -> d^2f/dx^2
\makeatother

%i-ter Ableitungsoperator
\newcommand{\dd}[2]{\ds\frac{\mathrm{d}^{#2}}{\mathrm{d}{#1}^{#2}}\,}   %\dd{x}{5} -> d^5/dx^5
\newcommand{\pp}[2]{\ds\frac{\partial^{#2}}{\partial{#1}^{#2}}\,}       %\pp{x}{5} -> d^5/dx^5 (partiell)

%----------------------Buchstaben, Räume----------------------
\def\eps{\varepsilon}
\def\N{\mathbb{N}}  %nat. Zahlen
\def\Z{\mathbb{Z}}  %ganze Zahlen
\def\Q{\mathbb{Q}}  %rat. Zahlen
\def\R{\mathbb{R}}  %reelle Zahlen
\def\C{\mathbb{C}}  %komplexe Zahlen
\def\P{\mathcal{P}} %Potenzmenge, Polynome
\def\T{\mathcal{T}} %Triangulierung
\def\X{\mathcal{T}} %samping point space
\def\Oe{\overset{..}{O}}    %Menge von 2013_12_04
\def\DD{\mathcal{D}} % Differentialoperator

\renewcommand{\i}[2]{\ds\int\limits_{#1}^{#2}} %Integral, %TODO_Lorin:das überschreibt "interpolierende" \I, %FIX_Benni: zweimal kleiner Buchstabe (Großbuchstaben sind eher für Räume)
\renewcommand{\s}[2]{\ds\sum\limits_{#1}^{#2}} %Summe %EDIT_Georg: mit renewcommand hat's nicht compiliert, deshalb jetzt newcommand


\renewcommand{\O}{\mathcal{O}}      %O-Notation
\renewcommand{\o}{o}
\newcommand{\CC}{\mathcal{C}}       %Raum der stetig diff.baren Fkt
\renewcommand{\L}{\mathcal{L}}      %Raum der Lebesgue-int.baren Fkt
\newcommand{\W}{\mathcal{W}}
\newcommand{\Lloc}{\L^1_{\text{loc}}}
\newcommand{\Cabh}{\mathrm{C}}      %Abhängigkeitskegel
\newcommand{\Sabh}{\mathrm{S}}      %zum Abhängigkeitskegel gehörendes S

% Maßeinheiten
\newcommand{\cm}{\,\mathrm{cm}}
\newcommand{\m}{\,\mathrm{m}}
\newcommand{\Npcm}{\,\mathrm{N/cm}}
\newcommand{\Npm}{\,\mathrm{N/m}}
\newcommand{\Npmm}{\,\mathrm{N/m^2}}
\newcommand{\NN}{\,\mathrm{N}}

%---------------------fette Buchstaben------------------------
\newcommand{\bfa}{\textbf{a}}
\newcommand{\bfb}{\textbf{b}}
\newcommand{\bfc}{\textbf{c}}
\newcommand{\bfd}{\textbf{d}}
\newcommand{\bfe}{\textbf{e}}
\newcommand{\bff}{\textbf{f}}
\newcommand{\bfg}{\textbf{g}}
\newcommand{\bfh}{\textbf{h}}
\newcommand{\bfi}{\textbf{i}}
\newcommand{\bfj}{\textbf{j}}
\newcommand{\bfk}{\textbf{k}}
\newcommand{\bfl}{\textbf{l}}
\newcommand{\bfm}{\textbf{m}}
\newcommand{\bfn}{\textbf{n}}
\newcommand{\bfo}{\textbf{o}}
\newcommand{\bfp}{\textbf{p}}
\newcommand{\bfq}{\textbf{q}}
\newcommand{\bfr}{\textbf{r}}
\newcommand{\bfs}{\textbf{s}}
\newcommand{\bft}{\textbf{t}}
\newcommand{\bfu}{\textbf{u}}
\newcommand{\bfv}{\textbf{v}}
\newcommand{\bfw}{\textbf{w}}
\newcommand{\bfx}{\textbf{x}}
\newcommand{\bfy}{\textbf{y}}
\newcommand{\bfz}{\textbf{z}}
\newcommand{\bfA}{\textbf{A}}
\newcommand{\bfB}{\textbf{B}}
\newcommand{\bfC}{\textbf{C}}
\newcommand{\bfD}{\textbf{D}}
\newcommand{\bfE}{\textbf{E}}
\newcommand{\bfF}{\textbf{F}}
\newcommand{\bfG}{\textbf{G}}
\newcommand{\bfH}{\textbf{H}}
\newcommand{\bfI}{\textbf{I}}
\newcommand{\bfJ}{\textbf{J}}
\newcommand{\bfK}{\textbf{K}}
\newcommand{\bfL}{\textbf{L}}
\newcommand{\bfM}{\textbf{M}}
\newcommand{\bfN}{\textbf{N}}
\newcommand{\bfO}{\textbf{O}}
\newcommand{\bfP}{\textbf{P}}
\newcommand{\bfQ}{\textbf{Q}}
\newcommand{\bfR}{\textbf{R}}
\newcommand{\bfS}{\textbf{S}}
\newcommand{\bfT}{\textbf{T}}
\newcommand{\bfU}{\textbf{U}}
\newcommand{\bfV}{\textbf{V}}
\newcommand{\bfW}{\textbf{W}}
\newcommand{\bfX}{\textbf{X}}
\newcommand{\bfY}{\textbf{Y}}
\newcommand{\bfZ}{\textbf{Z}}
\newcommand{\bfzero}{\textbf{0}}

\newcommand{\bfalpha}{\boldsymbol{\alpha}}
\newcommand{\bfbeta}{\boldsymbol{\beta}}
\newcommand{\bfgamma}{\boldsymbol{\gamma}}
\newcommand{\bfdelta}{\boldsymbol{\delta}}
\newcommand{\bfeps}{\boldsymbol{\eps}}
\newcommand{\bftheta}{\boldsymbol{\theta}}
\newcommand{\bfeta}{\boldsymbol{\eta}}
\newcommand{\bfphi}{\boldsymbol{\phi}}
\newcommand{\bfvarphi}{\boldsymbol{\varphi}}
\newcommand{\bfpsi}{\boldsymbol{\psi}}
\newcommand{\bfsigma}{\boldsymbol{\sigma}}
\newcommand{\bfPi}{\boldsymbol{\Pi}}
\newcommand{\bfXi}{\boldsymbol{\Xi}}
\newcommand{\bfxi}{\boldsymbol{\xi}}
\newcommand{\bfchi}{\boldsymbol{\chi}}
\newcommand{\bfChi}{\boldsymbol{\chi}}
\newcommand{\bfzeta}{\boldsymbol{\zeta}}
\newcommand{\bfmu}{\boldsymbol{\mu}}
\newcommand{\bflambda}{\boldsymbol{\lambda}}
\newcommand{\bfomega}{\boldsymbol{\omega}}
