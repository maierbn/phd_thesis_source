\section{Model of Muscle Contraction}\label{sec:model_muscle_contraction}


%Because of possibly large strains, a nonlinear hyperelastic formulation is used. 
%For mathematical foundations in continuum mechanics, we refer to basic literature such as the books of Holzapfel \cite{holzapfel2000nonlinear} and Marsden and Hughes \cite{marsden1994mathematical}, as well as literature on the application of the finite element method in continuum mechanics \cite{zienkiewicz1977finite,SUSSMAN1987357,zienkiewicz2005finite}.

%The following section provides a more profound introduction of solid mechanics to complement the overview given in \cref{sec:model_muscle_contraction}. It also describes the finite element discretization of the solid mechanics model and the algorithms used to obtain a numeric solution. 

Muscle contraction is described on the organ level by a solid mechanics model. The goal is to describe the deformation of the tissue caused by the internal forces that are generated by sarcomeres and as a response to outer constraints such as applied forces, the attachment to tendons and inertia effects.

Different modeling approaches exist to describe the mechanical muscle behavior. Dynamic \emph{finite elasticity} methods for large strains exist that use hyperelastic materials, both compressible and incompressible. Further, \emph{linear elasticity} descriptions with linearizations at various levels are used in appropriate applications where small strains can be assumed. The whole range from simplifying linearized models to accurate nonlinear approaches can be found in the literature, sometimes with varying conventions and symbols.
In this section, we introduce consistent notation and formulate the model equations for both approaches. The discretization and solution is discussed later in \cref{sec:discretization_mechanics}.

The derivation largely follows the book of Holzapfel \cite{holzapfel2000nonlinear} and the discretization follows the work of Zienkiewicz, Taylor et al. \cite{zienkiewicz1977finite,zienkiewicz2005finite}. Further details can be found also in the book of Marsden and Hughes \cite{marsden1994mathematical}.

%  The implementation of a solver for such generic descriptions exploiting parallel execution and integrating a multi-scale biomechanics model, being a contribution of this work, is an interdisciplinary endeavour. Therefore, we introduce consistent notation and summarize the required basics and the derivation up to the final algorithm  such that it may serve also readers that are not specialized in the field of continuum mechanics. 

% -v-

\subsection{Geometric Description}\label{sec:geometric_description}

% introduce quantities
% F, C, E, E(u), variation δE
%  S, P, sigma

We begin with the geometric description of the material body and define the basic quantities that are subsequently used to describe the physics. We consider the 3D muscle domain $\Omega_0=\Omega_M \subset \R^3$ in reference configuration at time $t=0$ that deforms into a current configuration $\Omega_t$ at time $t$. The material points are given by $\bfX \in \Omega_0$. The corresponding points $\bfx \in \Omega_t$ in the current configuration are defined by the function $\bfx = \bfvarphi_t(\bfX)$. 

In the following, capital letters refer to quantities in material or Lagrangian description, i.e., defined in the reference configuration and small letters refer to quantities in spatial or Eulerian description, i.e., defined in the current configuration.

The relation of point coordinates in the current configuration with respect to the reference configuration can also be described by the displacements field $\bfU$:
\begin{align*}
  \bfx(\bfX) = \bfX + \bfU(\bfX).
\end{align*}
The symbol $\bfu$ with $\bfu(\bfx(\bfX))=\bfU(\bfX)$ denotes the displacements formulated in current configuration.
The current velocity $\bfv$ is the time derivative of the displacements, $\bfv := \dot{\bfu}$.

The deformation gradient $\bfF$ is the second order tensor that is obtained by differentiating the function $\bfvarphi_t$. It is given using the unit vectors $\bfe_i$ and components $F_{aA}$:
\begin{align*}
  \bfF &= F_{aA}\,\bfe_a \otimes \bfe_A, \quad && F_{aA} = \p{x_a}{X_A}.
\end{align*}
Capital and small indices refer to reference and current configuration, respectively. The deformation gradient can also be expressed using the displacement field $\bfU$:
\begin{align}\label{eq:solid1}
  \bfF = \bfI + ∇\bfU.
\end{align}
%
Here and in the following, the gradient symbol $∇$ refers to differentiation with respect to material coordinates $\bfX$. 
We assume Cartesian coordinates.

% tangent, normal, volume map
The determinant of the deformation gradient is $J:= \det \bfF>0$. It is positive for any physically valid transformation.
The deformation gradient is used to map geometric quantities from the reference to the current configuration:
\begin{subequations}\label{eq:geometry_maps}
  \begin{align}
    \bft &= \bfF\,\bfT, & \text{(tangent map)} \label{eq:tangent_map}\\[4mm]
    \bfa &= \cof(\bfF)\,\bfA, & \text{(normal map)} \label{eq:normal_map}\\[4mm]
    v &= J\,V. & \text{(volume map)}\label{eq:volume_map}
  \end{align}
\end{subequations}
%
As given in \cref{eq:tangent_map} and visualized in \cref{fig:geometric_quantities}, the tensor $\bfF$ maps material tangents $\bfT$ in $\Omega_0$ to the corresponding spatial line elements $\bft$ in $\Omega_t$. 
Accordingly, the spatial stretch at a point $\bfx \in \Omega_t$ in a certain direction is given by $\lambda=\sqrt{\bflambda^\top\,\bflambda}$ with $\bflambda = \bfF\,\bfM$, where $\bfM$ is a material line element with unit length pointing in the respective Lagrangian direction.

In \cref{eq:normal_map}, the cofactor of $\bfF$ given by $\cof(\bfF) = J\,\bfF^{-\top}$ maps normals $\bfA$ and surface areas $|\bfA|$ from $\Omega_0$ to the corresponding values $\bfa$ and $|\bfa|$ in $\Omega_t$. Nanson's formula, $\d \bfa = \cof(\bfF)\,\d\bfA$, is used to transform surface integrals from Eulerian to Lagrangian description.
Note that tangents at a point $\bfX$ live in the tangent space $T_\bfX\Omega_0$ and normals live in the co-tangent space $T^\ast_\bfX\Omega_0$.

\Cref{eq:volume_map} describes the volume map from $\Omega_0$ to $\Omega_t$, which simply scales the reference volume $V$ by the determinant $J$ to obtain the volume $v$ in the current configuration.

% geometric quantities
\begin{figure}
  \centering%
  \def\svgwidth{0.7\textwidth}
  \input{images/theory/geometric_quantities.pdf_tex}%
  \caption{Vector spaces and variables used in the geometric description of the solid mechanics model. The left side shows the reference configuration with tangent and co-tangent space of point $\bfX$. The right side shows tangent and co-tangent space for the current domain and a point $\bfx$. The spatial stretch $\lambda$ is defined by mapping a material element $\bfM$ to the current configuration. The maps $\bfvarphi_t$, $\bfF$ and $\bfF^{-\top}$ map tangents $\bfT,\bft$ and normals $\bfA,\bfa$ between the configurations.}%
  \label{fig:geometric_quantities}%
\end{figure}

Furthermore, the deformation gradient $\bfF$ is used to define the right Cauchy Green tensor $\bfC = \bfF^\top\bfF$, which maps from tangent to co-tangent space in reference configuration, and subsequently the Green-Lagrange strain tensor:
%
\begin{align*}
  \bfE = \dfrac12(\bfC - \bfI).
\end{align*}
%
This strain measure can be interpreted as comparing the current Lagrangian metric $\bfC$, a measure for the symmetric part of the current deformation, with the reference metric which is the identity. Using \cref{eq:solid1}, the Green-Lagrange strain tensor can be formulated in terms of derivatives of the displacements:%
\begin{align}\label{eq:green_lagrange_u}
  \bfE &= \dfrac12\big((∇\bfU)^\top + ∇\bfU + ∇\bfU^\top ∇\bfU\big).
\end{align}

In case of small displacements, a simplification is to not distinguish between reference and current configuration.
The strain expression given in \cref{eq:green_lagrange_u} can be linearized by neglecting products of the derivatives and using the spatial displacements $\bfu$ instead of $\bfU$. As a result, the linearized strain tensor $\eps$ is given by:
\begin{align}\label{eq:linearized_helper3}
  \bfeps = \dfrac12\big((∇\bfu)^\top + ∇\bfu\big).
\end{align}
It can be used together with linear material models to derive a completely linear model.

\subsection{Stress Metrics}

Continuum mechanical models establish equations for the unknown displacement function $\bfu$ and its evolution in time via relations between stresses and strains. In the following, we introduce the required stress metrics.

% stress measures 
The Cauchy stress tensor $\bfsigma$ results from Euler's cut principle: we consider the mechanical action on an arbitrary, virtual cut out of the body in current configuration. The contact forces on the cut surface at a point $\bfx$ are described by the traction force $\bft$.
The traction vector acts on the current configuration and is a function of the position $\bfx \in \Omega_t$ and the local orientation of the cut given by the normal vector $\bfn$. Cauchy's theorem states that this relation is linear and can be described by the second order Cauchy stress  tensor $\bfsigma$:
\begin{align}\label{eq:cauchy_theorem}
  \bft = \bfsigma \cdot \bfn.
\end{align}

Thus, the Cauchy stress describes the \say{true stress} of contact forces per deformed surface area. Both slots of the second order tensor are associated with the current configuration. More specifically, $\bfsigma$ is contravariant and maps from a normal $\bfn$ in co-tangent space $T^\ast_\bfx\Omega_t$ to the traction $\bft$ in tangent space $T_\bfx\Omega_t$. 

While the physical description is natural in this Eulerian setting, the numerical treatment is more convenient in the Lagrangian setting, where we can integrate over a non-deforming domain. 
Moreover, a two-point setting, where surface areas are measured in the undeformed configuration and traction forces are measured in the deformed configuration, is often useful in engineering. This is the natural setting, e.g., in tension tests. Therefore, other stress measures involving the reference configuration are defined.

% numerics -> physics
% pull-back, push-back
Using the mappings presented in \cref{eq:geometry_maps}, all quantities can be transformed between both configurations. 
The physical derivation can be carried out equivalently in a Lagrangian or Eulerian setting and switching between them is possible at any point in the derivation. For this purpose, two operations are defined: the pull-back $\varphi^\ast(\bfa) = \bfF^\top\bfa\,\bfF$ and push-forward operations $\varphi_\ast(\bfA) = \bfF^{-\top}\bfA\,\bfF^{-1}$, which bring tensors from Eulerian to Lagrangian description and vice-versa.

% sigma, 1st PK, 2nd PK
The first Piola-Kirchhoff stress tensor $\bfP$ measures contact forces in the current configuration with regard to the area of the reference configuration and relates to the Cauchy stress as $\bfP = \bfsigma\,\cof(\bfF)$. The second Piola-Kirchhoff tensor $\bfS$ is a fully Lagrangian field given as the pull-back of the Cauchy stress scaled by $J$:%
\begin{align*}
  \bfS = \varphi^\ast(J\,\bfsigma) = J\,\bfF^{-1}\bfsigma\,\bfF^{-\top}.
\end{align*}
%It appears in the summary of the muscle contraction model in \cref{eq:contraction} in connection with the strain energy function $\Psi$.

% stress tensors
\begin{figure}
  \centering%
  \def\svgwidth{0.5\textwidth}
  \input{images/theory/stress_tensors.pdf_tex}%
  \caption{Stress tensors and geometric maps that can be used together in a solid mechanics formulation. The right Cauchy-Green tensor $\bfC$  and the second Piola-Kirchhoff stress $\bfS$ are dual Eulerian tensors and map between tangent space $T_\bfX\Omega_0$ and co-tangent space $T^\ast_\bfX\Omega_0$ in the reference domain. The deformation gradient $\bfF$ and the first Piola-Kirchhoff stress $\bfP$ are dual two-point tensors mapping from the reference to the current configuration. The Cauchy stress $\bfsigma$ is defined entirely in the Eulerian setting.
  %The Eulerian metric $\bfg$, which is the identity in cartesian coordinates and the Kirchhoff stress $J\,\bfsigma$ (where $\bfsigma$ is the Cauchy stress) are the dual objects in the Eulerian setting. All three pairs of dual tensors are linked together by transformations such as pull-back and push-forward.
  }%
  \label{fig:stress_tensors}%
\end{figure}

\Cref{fig:stress_tensors} summarizes the geometric maps by black arrows and the stress measures by red arrows. 
%To relate strains and stresses in a material model, the corresponding tensors have to be dual objects. Here, three settings are possible: The Eulerian setting uses the metric $\bfg$ and the dual Kirchhoff stress $J\,\bfsigma$. The two-point setting uses the deformation gradient $\bfF$ and the dual first Piola-Kirchhoff stress $\bfP$. The Lagrangian setting uses the right Cauchy-Green tensor $\bfC$ and the dual second Piola-Kirchhoff stress $\bfS$. 
%Different matching pairs of strain and stress tensors can be used to describe physical relations. 
The Lagrangian setting defines the right Cauchy-Green tensor $\bfC$ and the second Piola-Kirchhoff stress tensor $\bfS$.
We use these quantities in the derivation of the discretized equations, because the Lagrangian formulation is natural for this task and allows integrating over the non-deforming domain $\Omega_0$.

The Cauchy stress $\bfsigma$ is completely defined in the Eulerian setting. It is used to formulate physical balance principles.

% ---
\subsection{Overview of the Physical Relations}

The previously introduced quantities are linked together by various relations, which are summarized in the diagram in \cref{fig:tonti_diagram}. The goal is to find the relationship between given forces (top left in \cref{fig:tonti_diagram}) and the resulting deformation of the body described by the displacements (top right in \cref{fig:tonti_diagram}).
Prescribed external traction forces $\bfT$ and external or inertial body forces $\bfB$ act on the body and result in stresses $\bfS$ satisfying the \emph{equilibrium} relation. A \emph{material law} connects stresses $\bfS$ and strains $\bfE$. The \emph{kinematics} of the body determine the relationship between displacements $\bfu$ and strains $\bfE$. Geometric Dirichlet boundary conditions prescribe displacements and Neumann boundary conditions such as traction forces contribute to the stress field. 

Whereas the equilibrium relation is linear, the material and kinematic descriptions can both be chosen to be linear or nonlinear. 
In cases of small strains, geometric and material linearity can be assumed. We derive two such formulations: a linear formulation where all relations are linear and a nonlinear formulation with nonlinear material and kinematic relations.

% Tonti diagram
\begin{figure}
  \centering%
  \def\svgwidth{\textwidth}
  \input{images/theory/tonti_diagram.pdf_tex}%
  \caption{The three relations between various quantities that compose the solid mechanics model: Equilibrium links traction and body forces $\bfT$ and $\bfB$ to the stresses $\bfS$. A material model connects them to strains $\bfE$. The kinematic relations yield the resulting displacement field $\bfu$. Note that all quantities in this diagram are given in Lagrangian formulation.}%
  \label{fig:tonti_diagram}%
\end{figure}
% linear, static and dynamic

\subsection{Assumptions and Model Equations}\label{sec:assumptions_and_model_equations}
% ----

The foundation of continuum mechanics usually builds on three balance principles: conservation of mass, of momentum and of angular momentum. In the following, these principles are presented in their Eulerian forms.

First, we assume \emph{conservation of mass} in terms of the densities $\rho_0(\bfX)$ and $\rho(\bfx)$ in reference and current configurations:
%
\begin{align*}
  \ds\int\limits_{V_0} \rho_0\,\d V = \int\limits_{V_t} \rho \,\d v.
\end{align*}
%
The equation holds for all corresponding subdomains $V_0\subset \Omega_0$ and $V_t \subset \Omega_t$. With the intermediate step of deducing $\d/\d t \int_{\Omega_t} \rho \,\d v=0$, we get the following differential equation:%
\begin{align}\label{eq:contraction_helper1}
  \dot{\rho}(\bfv,t) + \rho(\bfx,t)\,\div\big(\bfv(\bfx,t)\big) = 0.
\end{align}
%

As muscle tissue largely consists of water, it is typically assumed to be an incompressible domain. This is equivalent to a constant density, $\dot{\rho}=0$, and, thus, \cref{eq:contraction_helper1} reduces to
%
\begin{align}\label{eq:assumption_1_local}
  \div(\bfv(\bfx,t)) = 0.
\end{align}
%

The second assumption is the \emph{balance of momentum}, which is expressed as %
\begin{align}\label{eq:assumption_2_integral}
  \d{t} \ds\int\limits_{V_t} \rho\,\bfv\, \d v = \ds\int\limits_{V_t} \rho\,\bfb \,\d v + \ds\int\limits_{∂V_t} \bft \,\d a.
\end{align}
Here, $\bfb$ describes a body force and $\bft$ describes a traction force that acts on the surface of the domain $V_t$. Using the Cauchy theorem \cref{eq:cauchy_theorem}, it can be replaced by the Cauchy stress $\bfsigma$. The corresponding differential form is given by the following differential equation:%
\begin{align}\label{eq:assumption_2_local}
  \rho\,\dot{\bfv}(\bfx,t) = \rho\,\bfb(\bfx,t) + \div\bfsigma(\bfx,t).
\end{align}
It relates external forces to the internal stress field and describes the \emph{equilibrium} relation in \cref{fig:tonti_diagram} in Eulerian form.
For the discretization, a Lagrangian form is typically used.


For hyperelastic materials, which we consider in the muscle model, the equilibrium relation can also be formulated in terms of the \emph{Hellinger-Reissner energy functional} $\Pi_L(\bfu,p)$, which describes the potential energy of the system depending on the displacement and pressure functions $\bfu$ and $p$. Analog to the local form in \cref{eq:assumption_2_local}, it contains terms for the external loads and for the internal response of the body.
The functional is additively composed of external and internal potential energy:
\begin{align}\label{eq:hellinger_reissner}
  \Pi_L(\bfu,p) = \Pi_\text{ext}(\bfu) + \Pi_\text{int}(\bfu,p).
\end{align}
The external energy functional is formulated by
\begin{align}\label{eq:pi_ext}
  \Pi_\text{ext}(\bfu) = -\ds\int_{\Omega_0} \bfB\, \bfu\,\d V - \ds\int_{∂\Omega_0^t}\bar{\bfT}\,\bfu\,\d S,
\end{align}
with body force $\bfB$ in reference configuration and prescribed surface traction $\bar{\bfT}$ on the traction boundary $∂\Omega_0^t$. The body force term $\bfB$ also includes the inertial forces of mass density times acceleration, $\rho\,\dot{\bfv}$, in case of a dynamic model.
The internal energy functional $\Pi_\text{int}(\bfu,p)$ describes the strain energy of the system depending on the displacement field $\bfu$ and the hydrostatic pressure $p$. The term is defined in \cref{sec:section_with_pi_int}.

The \emph{principle of stationary potential energy} demands that the potential energy functional $\Pi_L$ is stationary.
Variational calculus and differentiation of \cref{eq:hellinger_reissner} lead to the local Eulerian description given in \cref{eq:assumption_2_integral,eq:assumption_2_local}.


The third assumption is the \emph{balance of angular momentum} and can be formulated using the 3D cross-product:%
\begin{align*}
  \d{t} \ds\int\limits_{V_t} \bfx \times (\rho\,\bfv)\, \d v = \ds\int\limits_{V_t} \bfx \times (\rho\,\bfb) \,\d v + \ds\int\limits_{∂V_t} \bfx \times \bft\,\d a.
\end{align*}
%
This can be shown to be equivalent to the symmetry of the Cauchy stress tensor, $\bfsigma = \bfsigma^\top$.

A further assumption in the multi-scale muscle framework is to only consider isothermal conditions. 
An activated muscle performs work and energy is added to the system by metabolism. Further, the muscle is not thermodynamically isolated. The system is not closed regarding conversion and transfer of energy and, thus, the balance of energy cannot be modeled easily.

Regarding the required relations to obtain the deformation of the body from external loads given in \cref{fig:tonti_diagram}, the \emph{equilibrium} relation is given by \cref{eq:assumption_2_local} and the nonlinear \emph{kinematic} relation is given by \cref{eq:green_lagrange_u}.
The \emph{material} relation has yet to be defined. The mathematical description is closed by defining a constitutive relation between stresses and strains in the next sections.

\Cref{sec:material_linear_model} defines a linear material model that can be used together with linearized kinematics to formulate a fully linear model. \Cref{sec:material_nonlinear_model} presents the nonlinear material model to proceed with the fully nonlinear description. 

\subsection{Linear Material Model}\label{sec:material_linear_model}

For a linear constitutive relation between strain and stress, the linearized strain tensor $\bfeps$, defined in \cref{eq:linearized_helper3} is used together with the Eulerian Cauchy stress $\bfsigma$. The generic linear material model is \emph{Hooke}'s law, given by 
%
\begin{align}\label{eq:linearized_helper2}
  \bfsigma = \C:\bfeps
\end{align}
with the fourth order material tensor%
\begin{align*}
  \C_{abcd} = K\,δ_{ab}\,δ_{cd} + μ\,(δ_{ac}\,δ_{bd} + δ_{ad}\,δ_{bc} - \dfrac23 δ_{ab}\,δ_{cd}).
\end{align*}
The bulk modulus $K$ is a measure for the (in-)compressibility and the shear modulus $\mu$ specifies the elastic shear stiffness. $δ_{ab}$ is the Kronecker delta.
The material tensor $\C$ exhibits the following major and minor symmetries:%
\begin{subequations}\label{eq:symmetries}
\begin{align}
  \C_{abcd} &= \C_{cdab}, \quad &\text{(major symmetries)}\\[4mm]
  \C_{abcd} &= \C_{bacd} = \C_{abdc} = \C_{badc}, \quad & \text{(minor symmetries)}
\end{align}
\end{subequations}
effectively reducing the number of independent entries from 81 to 21 for 3D domains.

To incorporate force generation in the muscle, the stress can be additively composed of the passive stress $\bfsigma$ and an additional active stress term $\bfsigma_\text{active}$:%
\begin{align}\label{eq:active_stress_linear}
  \bfsigma^\text{total} = \bfsigma + \bfsigma^\text{active}.
\end{align}

\subsection{Nonlinear Material Modeling}\label{sec:material_modeling}
%
Next, we present the derivation of a nonlinear model that does not make any linearization assumptions of small strains as in the previous section.
We begin with the description of the material law, which links strains and stresses.

The scalar strain energy function $\Psi$ describes the elastic energy of the material depending on the deformation.
The definition of $\Psi$ suffices to describe the behavior of a hyperelastic material.
The strain energy function links the right Cauchy Green tensor $\bfC$ to the second Piola-Kirchhoff stress tensor $\bfS$ by the relation%
\begin{align}\label{eq:material_model_helper1}
  \bfS = 2\p{\Psi(\bfC)}{\bfC}.
\end{align}

The \emph{principle of material objectivity} requires that material properties are invariant under a change of observer. As a result, the \emph{representation theorem for isotropic materials} states that the stress tensor can be represented using three strain invariants $I_1, I_2$ and $I_3$. For a transversely isotropic material, two invariants $I_4$ and $I_5$ that depend on the anisotropy direction $\bfa_0$ (corresponding to a fiber direction) are added.
Consequently, we can formulate the strain energy function $\Psi=\Psi(I_1,I_2,I_3,I_4,I_5)$ in terms of these invariants. The principle strain invariants $I_1$ to $I_3$ of the right Cauchy-Green tensor $\bfC$ and the additional anisotropic invariants $I_4$ and $I_5$ are defined as:
\begin{align*}
  &I_1(\bfC) = \tr(\bfC),  &
  &I_2(\bfC) = \dfrac12\big(\tr(\bfC)^2 - \tr(\bfC^2)\big), &
  I_3(\bfC) = \det(\bfC) = J^2,\\[4mm]
  &I_4(\bfC,\bfa_0) = \bfa_0 \cdot \bfC \, \bfa_0, &
  &I_5(\bfC,\bfa_0) = \bfa_0 \cdot \bfC^2 \, \bfa_0. &
\end{align*}
The fiber stretch is related to the fourth invariant by $\lambda_f = \sqrt{I_4}$. Note that requiring incompressibility is equivalent to enforcing $J=1$, and, in this case, we get ${I_3(\bfC) = 1}$. 

It is convenient to use a decoupled description, where the deformation gradient $\bfF$ and the right Cauchy-Green tensor $\bfC$ are multiplicatively decomposed into volume-changing (volumetric) and volume-preserving (isochoric) parts:%
\begin{align*}
  \bfF &= (J^{1/3}\bfI)\,\bar{\bfF},  & \bfC &= (J^{2/3}\bfI)\,\bar{\bfC}.
\end{align*}
%
Here, the volumetric parts are the identity tensors scaled by a power of the determinant $J$ of the deformation gradient. The isochoric or distortional parts $\bar{\bfF}$ and $\bar{\bfC}$ are given by%
\begin{align}\label{eq:reduced_fc}
  \bar{\bfF} &= J^{-1/3}\,\bfF,  & \bar\bfC &= J^{-2/3}\,\bfC.
\end{align}
The reduced invariants $\bar{I}_1$ to $\bar{I}_5$ of the reduced right Cauchy-Green tensor $\bar\bfC$ are defined accordingly.
Similarly, the strain energy function has a decoupled representation with volumetric part $\Psi_\text{vol}$ and isochoric part $\Psi_\text{iso}$:
\begin{align}\label{eq:psi_iso}
  \Psi = \Psi_\text{vol}(J) + \Psi_\text{iso}(\bar{\bfC}) = \Psi_\text{vol}(J) + \Psi_\text{iso}(\bar{I}_1,\bar{I}_2,\bar{I}_4,\bar{I}_5).
\end{align}

Using the decoupled form, any incompressible material can be modeled with the \emph{penalty method} as follows. 
The material behavior is given by the isochoric strain energy $\Psi_\text{iso}(\bar{\bfC})$, e.g., by employing the Mooney-Rivlin model in \cref{eq:mooney_rivlin}. The volumetric part is defined as
\begin{align*}
  \Psi_\text{vol}(J) &= \kappa\,G(J) \qquad \text{with } G(J) = \dfrac12 (J-1)^2,
\end{align*}
with the incompressibility parameter $\kappa$ and the penalty function $G(J)$. This function is strictly convex and approaches zero as $J$ approaches 1. For large values of $\kappa$, the behavior is nearly incompressible. A disadvantage of this method is, that the resulting system becomes singular for $J \to 1$.

A better approach in this regard is to use a mixed formulation, where incompressibility is enforced exactly using a Lagrange multiplier. This approach is also implemented in OpenDiHu and is the preferred method for incompressible materials. 

In OpenDiHu, the strain energy function of a new material can be given using the following four terms:
%
\begin{align}\label{eq:definition_psi}
  \Psi = \Psi_\text{vol}(J) + \Psi_\text{iso}(\bar{I}_1,\bar{I}_2,\bar{I}_4,\bar{I}_5) + \Psi_1(I_1,I_2,I_3) + \Psi_2(\bfC,\bfa_0).
\end{align}
The decoupled form is available with $\Psi_\text{vol}$ and $\Psi_\text{iso}$, the coupled form for isotropic materials can be used via $\Psi_1$. The term $\Psi_2$ gives the most flexibility, as the constitutive model can be directly formulated using the right Cauchy-Green tensor $\bfC$ and the fiber direction $\bfa_0$. The unused terms among $\Psi_\text{vol},\Psi_\text{iso},\Psi_1$ and $\Psi_2$ can be defined as constant zero. The incompressibility constraint using Lagrange multipliers can be switched on or off such that both incompressible and compressible materials can be computed.
%
% ---

\subsection{The Nonlinear Material Model for Muscle Contraction}\label{sec:material_nonlinear_model}

In the muscle contraction model of \cite{Heidlauf2013}, the strain energy function is additively composed of two passive terms, one isotropic, one anisotropic, and one additional active term:
\begin{align}\label{eq:transiso_mooney_rivlin}
  \Psi(\bfC) = \Psi_\text{isotropic}(I_1,I_2) + \Psi_\text{anisotropic}(\lambda_f) + \Psi_\text{active}(\gamma).
\end{align}
The isotropic term $\Psi_\text{isotropic}$ is formulated in terms of the strain invariants $I_1=\tr(\bfC)$ and $I_2=\big(\tr(\bfC)^2 - \tr(\bfC^2)\big)/2$. The anisotropic term $\Psi_\text{anisotropic}$ depends on the fiber stretch $\lambda_f$. The active term $\Psi_\text{active}$ yields the active stress that results from muscular activation, which is described by the activation parameter $\gamma$.
%Note the missing dependency on the third invariant $I_3$, which is constant because of the enforced incompressibility.

The passive behavior of muscle tissue is modeled by a transversely isotropic Mooney-Rivlin material.
The isotropic part is given by the Mooney-Rivlin formulation:%
\begin{align}\label{eq:mooney_rivlin}
  \Psi_\text{isotropic}(I_1,I_2) = c_1\,(I_1 - 3) + c_2\,(I_2-3).
\end{align}
The values of the two material parameters $c_1$ and $c_2$ can be determined by compression tests and are summarized in the work of \cite{Heidlauf2013}.

The anisotropic behavior depends only on the fiber stretch $\lambda_f$. The formulation in \cite{Heidlauf2013} uses two material parameters $b$ and $d$ and the following function:
\begin{align*}
  \Psi_\text{anisotropic}(\lambda_f) = \dfrac{b}{d}(\lambda_f^d - 1) - b\,\log(\lambda_f).
\end{align*}
%

The active contribution is directly formulated in terms of the second Piola-Kirchhoff stress $\bfS$. The relation between the active stress $\bfS_\text{active}$ and the active contribution $\Psi_\text{active}$ of the strain energy function as well as the definition of $\bfS_\text{active}$ is given as follows:
\begin{align}\label{eq:active_stress_term}
  \bfS_\text{active} = \dfrac{1}{\lambda_f}\p{\Psi_\text{active}}{\lambda_f} \bfA \otimes \bfA = \dfrac{1}{\lambda_f} \cdot S_\text{max,active}\cdot f_\ell(\lambda_f)\cdot\bar{\gamma}\, \bfA \otimes \bfA.
\end{align}
%
Here, the resulting active stress tensor $\bfS_\text{active}$ is the second order tensor oriented according to the material fiber direction $\bfA: \Omega_0 \to \R^3$ and given by the dyadic product $\bfA \otimes \bfA = A_{i}\,A_{j}\,\bfe_i \otimes \bfe_j$, scaled by the maximum active stress parameter $S_\text{max,active}$, a function $f_\ell$ that models the force-length relation, and the 3D homogenized value $\bar{\gamma}$ of the activation parameter $\gamma \in [0,1]$ following from the half-sarcomere model.

In the deforming body fat layer, the active stress contribution is disregarded. For simulating tendons, different material models can be used such as the model proposed by Carniel et al. \cite{Carniel2017}, which describes microstructural interactions between collagen fibers and their matrix in addition to the elastic response of the fibers themselves. To alter the material model, the definition of $\Psi$ can simply be changed while all other equations of the solid mechanics model remain intact. 
%Similarly, other material models can be defined using the framework of the strain energy function.

\subsection{Summary of the Solid Mechanics Model Equations}
In summary, the model of solid mechanics for muscle contraction is solved for the unknown displacements $\bfu$ and additionally the velocities $\bfv$ if a dynamic formulation is considered.

The model equations follow from the following balance principles:
%
\begin{subequations}\label{eq:contraction}
  \begin{align}
    \div(\bfv) &= 0, \qquad &&\text{(incompressibility)} \label{eq:contraction_1}\\[4mm]
    \rho\,\dot{\bfv} &= \rho\,\bfb + \div\bfsigma, && \text{(balance of linear momentum)}\label{eq:contraction_2}\\[4mm]
    \bfsigma &= \bfsigma^\top, && \text{(balance of angular momentum)}\label{eq:contraction_3}
  \end{align}
\end{subequations}
with the constant density $\rho$, external body forces $\bfb$ and the Cauchy stress tensor $\bfsigma$.

Additionally, geometric relations between displacements $\bfu$ and strains $\bfE$ or $\bfeps$ are assumed, either fully nonlinear in \cref{eq:green_lagrange_u} or with corresponding linearization assumptions in \cref{eq:linearized_helper3}.
Furthermore, a material model is given that relates strains and stresses. A linear model is described in \cref{sec:material_linear_model}. The framework for nonlinear hyperelastic models uses a strain energy function $\Psi$ as described in \cref{sec:material_modeling}. A particular nonlinear material model for muscle contraction from the literature is described in \cref{sec:material_nonlinear_model}.

The description of the multi-scale model \cite{Roehrle2012,Heidlauf2013} assumes quasi-static conditions, which means that the velocities are set to zero, $\bfv=\bfzero$, and inertial terms are neglected. As a consequence, the incompressibility constraint in \cref{eq:contraction_1} has to be formulated differently and the balance of momentum in \cref{eq:contraction_2} reduces to $\rho\,\bfb + \div \bfsigma = 0$.
Our implementation extends the model to the fully dynamic formulation given in \cref{eq:contraction_1,eq:contraction_2,eq:contraction_3}. 

Initial conditions for the displacements $\bfu$ and velocities $\bfv$ define the initial pose of the muscle tissue:
%
\begin{align*}
  \bfu(\bfx,0) &= \bfu_0(\bfx), & \bfv(\bfx,0) &= \bfv_0(\bfx) \quad &&\text{for } \bfx \in \Omega_M.
\end{align*}
%
Dirichlet boundary conditions for $\bfu$ and $\bfv$ can fix certain parts of the muscle, e.g., at the attachment points of the tendons:
\begin{align*}
  \bfu(\bfx,t) &= \bar{\bfu}(t), & \bfv(\bfx,t) &= \bar{\bfv}(t) \quad &&\text{for } \bfx \in ∂\Omega_\text{Dirichlet}.
\end{align*}
%
Additionally, Neumann boundary conditions can be used to prescribe traction forces on the surface.

The derivation of the finite element formulation and the resulting numerical scheme to obtain the solution functions $\bfu$ and $\bfv$ are discussed in \cref{sec:discretization_mechanics}.

%\subsection{Sensory Organs and Motor Neurons}
%text of paper:
%Muscle spindles and Golgi tendon organs are located in the muscle and sense fiber stretch, contraction velocity, contraction acceleration and muscle forces. 
%They are connected via layers of inter neurons to the motor neurons, which reside in the spinal cord.
%In turn, the motor units innervate and activate the muscle.

%We use CellML models to compute the dynamics of muscle spindles \cite{Mileusnic2006} and motor neurons \cite{CisiKohn2008}.

